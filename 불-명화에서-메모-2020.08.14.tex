%	-------------------------------------------------------------------------------
%
%		작성		2020년 
%				7월 
%				20일 
%				월
%
%				구덕 도서관 654.22 10
%
%
%
%
%	-------------------------------------------------------------------------------

%\documentclass[10pt,xcolor=pdftex,dvipsnames,table]{beamer}
%\documentclass[10pt,blue,xcolor=pdftex,dvipsnames,table,handout]{beamer}
%\documentclass[14pt,blue,xcolor=pdftex,dvipsnames,table,handout]{beamer}
\documentclass[aspectratio=1610,12pt,xcolor=pdftex,dvipsnames,table,handout]{beamer}

		% Font Size
		%	default font size : 11 pt
		%	8,9,10,11,12,14,17,20
		%
		% 	put frame titles 
		% 		1) 	slideatop
		%		2) 	slide centered
		%
		%	navigation bar
		% 		1)	compress
		%		2)	uncompressed
		%
		%	Color
		%		1) blue
		%		2) red
		%		3) brown
		%		4) black and white	
		%
		%	Output
		%		1)  	[default]	
		%		2)	[handout]		for PDF handouts
		%		3) 	[trans]		for PDF transparency
		%		4)	[notes=hide/show/only]

		%	Text and Math Font
		% 		1)	[sans]
		% 		2)	[sefif]
		%		3) 	[mathsans]
		%		4)	[mathserif]


		%	---------------------------------------------------------	
		%	슬라이드 크기 설정 ( 128mm X 96mm )
		%	---------------------------------------------------------	
%			\setbeamersize{text margin left=2mm}
%			\setbeamersize{text margin right=2mm}

	%	========================================================== 	Package
		\usepackage{kotex}						% 한글 사용
		\usepackage{amssymb,amsfonts,amsmath}	% 수학 수식 사용
		\usepackage{color}					%
		\usepackage{colortbl}					%


		\usepackage{minitoc}					%

	%		========================================================= 	note 옵션인 
	%			\setbeameroption{show only notes}
		

	%		========================================================= 	Theme

		%	---------------------------------------------------------	
		%	전체 테마
		%	---------------------------------------------------------	
		%	테마 명명의 관례 : 도시 이름
%			\usetheme{default}			%
%			\usetheme{Madrid}    		%
%			\usetheme{CambridgeUS}    	% -red, no navigation bar
%			\usetheme{Antibes}			% -blueish, tree-like navigation bar

		%	----------------- table of contents in sidebar
			\usetheme{Berkeley}		% -blueish, table of contents in sidebar
									% 개인적으로 마음에 듬

%			\usetheme{Marburg}			% - sidebar on the right
%			\usetheme{Hannover}		% 왼쪽에 마크
%			\usetheme{Berlin}			% - navigation bar in the headline
%			\usetheme{Szeged}			% - navigation bar in the headline, horizontal lines
%			\usetheme{Malmoe}			% - section/subsection in the headline

%			\usetheme{Singapore}
%			\usetheme{Amsterdam}

		%	---------------------------------------------------------	
		%	색 테마
		%	---------------------------------------------------------	
%			\usecolortheme{albatross}	% 바탕 파란
%			\usecolortheme{crane}		% 바탕 흰색
%			\usecolortheme{beetle}		% 바탕 회색
%			\usecolortheme{dove}		% 전체적으로 흰색
%			\usecolortheme{fly}		% 전체적으로 회색
%			\usecolortheme{seagull}	% 휜색
%			\usecolortheme{wolverine}	& 제목이 노란색
%			\usecolortheme{beaver}

		%	---------------------------------------------------------	
		%	Inner Color Theme 			내부 색 테마 ( 블록의 색 )
		%	---------------------------------------------------------	

%			\usecolortheme{rose}		% 흰색
%			\usecolortheme{lily}		% 색 안 칠한다
%			\usecolortheme{orchid} 	% 진하게

		%	---------------------------------------------------------	
		%	Outter Color Theme 		외부 색 테마 ( 머리말, 고리말, 사이드바 )
		%	---------------------------------------------------------	

%			\usecolortheme{whale}		% 진하다
%			\usecolortheme{dolphin}	% 중간
%			\usecolortheme{seahorse}	% 연하다

		%	---------------------------------------------------------	
		%	Font Theme 				폰트 테마
		%	---------------------------------------------------------	
%			\usfonttheme{default}		
			\usefonttheme{serif}			
%			\usefonttheme{structurebold}			
%			\usefonttheme{structureitalicserif}			
%			\usefonttheme{structuresmallcapsserif}			



		%	---------------------------------------------------------	
		%	Inner Theme 				
		%	---------------------------------------------------------	

%			\useinnertheme{default}
			\useinnertheme{circles}		% 원문자			
%			\useinnertheme{rectangles}		% 사각문자			
%			\useinnertheme{rounded}			% 깨어짐
%			\useinnertheme{inmargin}			




		%	---------------------------------------------------------	
		%	이동 단추 삭제
		%	---------------------------------------------------------	
%			\setbeamertemplate{navigation symbols}{}

		%	---------------------------------------------------------	
		%	문서 정보 표시 꼬리말 적용
		%	---------------------------------------------------------	
%			\useoutertheme{infolines}


			
	%	---------------------------------------------------------- 	배경이미지 지정
%			\pgfdeclareimage[width=\paperwidth,height=\paperheight]{bgimage}{./fig/Chrysanthemum.jpg}
%			\setbeamertemplate{background canvas}{\pgfuseimage{bgimage}}

		%	---------------------------------------------------------	
		% 	본문 글꼴색 지정
		%	---------------------------------------------------------	
%			\setbeamercolor{normal text}{fg=purple}
%			\setbeamercolor{normal text}{fg=red!80}	% 숫자는 투명도 표시


		%	---------------------------------------------------------	
		%	itemize 모양 설정
		%	---------------------------------------------------------	
%			\setbeamertemplate{items}[ball]
%			\setbeamertemplate{items}[circle]
%			\setbeamertemplate{items}[rectangle]






		\setbeamercovered{dynamic}





		% --------------------------------- 	문서 기본 사항 설정
		\setcounter{secnumdepth}{3} 		% 문단 번호 깊이
		\setcounter{tocdepth}{3} 			% 문단 번호 깊이




% ------------------------------------------------------------------------------
% Begin document (Content goes below)
% ------------------------------------------------------------------------------
	\begin{document}
	

			\title{ 명화에서 길을 찾다 }
			\author{ 김대희 }
			\date{ 2020년 
					08월 
					14일 
					금요일 }


% -----------------------------------------------------------------------------
%		개정 내용
% -----------------------------------------------------------------------------
%
%		2020년 7월 20일 첫제작
%
%
%


	%	==========================================================
	%
	%	----------------------------------------------------------
		\begin{frame}[plain]
		\titlepage
		\end{frame}


		\begin{frame} [plain]{목차}
		\tableofcontents%
		\end{frame}



	%	========================================================== 개요
		\part{개요}
		\frame{\partpage}

		\begin{frame} [plain]{목차}
		\tableofcontents%
		\end{frame}
		

	%	---------------------------------------------------------- 지은이
	%		Frame
	%	----------------------------------------------------------
		\section{지은이}
		\begin{frame} [t,plain]
		\frametitle{지은이}
			\begin{block} {지은이}
			\setlength{\leftmargini}{5em}			
			\begin{itemize}
				\item [지은이] 강소연
			\end{itemize}
			\end{block}						

		\end{frame}						
		

	%	---------------------------------------------------------- 도서
	%		Frame
	%	----------------------------------------------------------
		\section{도서}
		\begin{frame} [t,plain]
		\frametitle{도서}
			\begin{block} {도서}
			\setlength{\leftmargini}{5em}			
			\begin{itemize}
				\item [중앙] 654.22-23
				\item [수정] 652.22-12
				\item [구덕] 654.22 10
				\item [남구] 654.22-강55
			\end{itemize}
			\end{block}			

								
		\end{frame}						
	

	%	---------------------------------------------------------- 제목 
	%		Frame
	%	----------------------------------------------------------
		\section{ 제목 }
		\begin{frame} [t,plain]
		\frametitle{ 제목 }
			\begin{block} { 제목 }
			\setlength{\leftmargini}{5em}			
			\begin{itemize}
				\item 매혹적인 우리 불화 속 지혜
				\item 명화에서 길을 찾다
			\end{itemize}
			\end{block}			

								
		\end{frame}						

	%	========================================================== 내용
		\part{ 내용 }
		\frame{\partpage}

		\begin{frame} [plain]{목차}
		\tableofcontents%
		\end{frame}



	%	---------------------------------------------------------- 목차
	%		Frame
	%	----------------------------------------------------------
		\section{ 목차 }
		\begin{frame} [t,plain]
		\frametitle{ 목차 }
			\begin{block} { 목차 }
			\setlength{\leftmargini}{2em}			
			\begin{itemize}
				\item 서문
				\item 그림과 경전 소개
				\item 1장 희생
				\item 2장 결심
				\item 3장 정진
				\item 4장 평온
				\item 5장 인내
				\item 6장 욕망
				\item 7장 지혜
				\item 8장 선정
				\item 9장 방편
				\item 10장 자비
			\end{itemize}
			\end{block}						
								
		\end{frame}						

	%	---------------------------------------------------------- 서문
	%		Frame
	%	----------------------------------------------------------
		\section{ 서문 }
		\begin{frame} [t,plain]
		\frametitle{ 서문 }
			\begin{block} { 서문 }
			\setlength{\leftmargini}{2em}			
			\begin{itemize}
				\item 
			\end{itemize}
			\end{block}						
								
		\end{frame}						

	%	---------------------------------------------------------- 그림과 경전 소개
	%		Frame
	%	----------------------------------------------------------
		\section{ 그림과 경전 소개 }
		\begin{frame} [t,plain]
		\frametitle{ 그림과 경전 소개 }
			\begin{block} { 그림과 경전 소개 }
			\setlength{\leftmargini}{2em}			
			\begin{itemize}
				\item 
			\end{itemize}
			\end{block}						
								
		\end{frame}						


		

	%	========================================================== 내용
		\part{ 내용 }
		\frame{\partpage}
		
		\begin{frame} [plain]{목차}
		\tableofcontents%
		\end{frame}
		

% -----------------------------------------------------------------------------
%
% -----------------------------------------------------------------------------
	\section{	1장	희생	:	안락국 태자 변상도		}
	\frame [plain] {\sectionpage}

	%	---------------------------------------------------------- 
	%		Frame
	%	----------------------------------------------------------
		\begin{frame} [t,plain]
		\frametitle{ 희생 : 안락국태자 변상도  }
			\begin{block} { 희생: 안락국 태자 이야기  }
			\setlength{\leftmargini}{2em}			
			\begin{itemize}
				\item 
			\end{itemize}
			\end{block}						
								
		\end{frame}						


% -----------------------------------------------------------------------------
%
% -----------------------------------------------------------------------------
	\section{	2장	결심	:	지장사왕도		}
	\frame [plain] {\sectionpage}

	%	---------------------------------------------------------- 
	%		Frame
	%	----------------------------------------------------------
		\begin{frame} [t,plain]
		\frametitle{ 바라문의 딸 이야기 }
			\begin{block} { 바라문의 딸 이야기 }
			\setlength{\leftmargini}{2em}			
			\begin{itemize}
				\item 
			\end{itemize}
			\end{block}						
								
		\end{frame}						

	%	---------------------------------------------------------- 
	%		Frame
	%	----------------------------------------------------------
		\begin{frame} [t,plain]
		\frametitle{제1화  바라문의 딸 지옥으로 가다 }
			\begin{block} { 제1화 \\ 바라문의 딸 지옥으로 가다 }
			\setlength{\leftmargini}{2em}			
			\begin{itemize}
				\item 
			\end{itemize}
			\end{block}						
								
		\end{frame}						


	%	---------------------------------------------------------- 
	%		Frame
	%	----------------------------------------------------------
		\begin{frame} [t,plain]
		\frametitle{ 제2화 지옥 중생 모두 구하리! 큰세원을 세우다 }
			\begin{block} { 제2화 \\ 지옥 중생 모두 구하리!  큰세원을 세우다 }
%			\setlength{\leftargini}{2em}			
%			\begin{itemize}
%				\item 
%			\end{itemize}
			\end{block}						
								
		\end{frame}		

	%	---------------------------------------------------------- 
	%		Frame
	%	----------------------------------------------------------
		\begin{frame} [t,plain]
		\frametitle{ 제3화 인과응보의 원리로 중생을 깨우치다 }
			\begin{block} { 제3화 \\ 인과응보의 원리로 중생을 깨우치다 }
%			\setlength{\leftargini}{2em}			
%			\begin{itemize}
%				\item 
%			\end{itemize}
			\end{block}						
								
		\end{frame}		
				
% -----------------------------------------------------------------------------
%
% -----------------------------------------------------------------------------
	\section{	3장	정진	:	수월관음도		}
	\frame [plain] {\sectionpage}

% -----------------------------------------------------------------------------
%
% -----------------------------------------------------------------------------
	\section{	4장	평온	:	관경16관변상도		}
	\frame [plain] {\sectionpage}

% -----------------------------------------------------------------------------
%
% -----------------------------------------------------------------------------
	\section{	5장	인내	:	심우도		}
	\frame [plain] {\sectionpage}

% -----------------------------------------------------------------------------
%
% -----------------------------------------------------------------------------
	\section{	6장	욕망	:	감로도		}
	\frame [plain] {\sectionpage}

% -----------------------------------------------------------------------------
%
% -----------------------------------------------------------------------------
	\section{	7장	지혜	:	석가모니 팔상도		}
	\frame [plain] {\sectionpage}

% -----------------------------------------------------------------------------
%
% -----------------------------------------------------------------------------
	\section{	8장	선정	:	달마도		}
	\frame [plain] {\sectionpage}

% -----------------------------------------------------------------------------
%
% -----------------------------------------------------------------------------
	\section{	9장	방편	:	법화경변상도		}
	\frame [plain] {\sectionpage}

% -----------------------------------------------------------------------------
%
% -----------------------------------------------------------------------------
	\section{	10장	자비	:	관세음보살32응도		}
	\frame [plain] {\sectionpage}



	%	========================================================== 기도
		\part{기도}
		\frame{\partpage}

%		\begin{frame} [plain]{목차}
%		\tableofcontents%
%		\end{frame}


% -----------------------------------------------------------------------------
%
% -----------------------------------------------------------------------------
		\section{다라니기도}
		\begin{frame} [t,plain]
		\frametitle{다라니기도}
			\begin{block} {다라니기도 }
			\setlength{\leftmargini}{5em}			
			\begin{itemize}
				\item [장소]	 법당
				\item [일시]	 매주 수요일 저녁 8시
				\item [금액]	
			\end{itemize}
			\end{block}						
		\end{frame}					






	%	========================================================== 행사
		\part{행사}
		\frame{\partpage}
		
		\begin{frame} [plain]{목차}
		\tableofcontents%
		\end{frame}
		

	

	

	%	---------------------------------------------------------- 대중공양
	%		Frame
	%	----------------------------------------------------------
		\section{대중공양}
		\begin{frame} [t,plain]
		\frametitle{대중공양}
			\begin{block} {대중공양 }
			\setlength{\leftmargini}{5em}			
			\begin{itemize}
				\item [장소]	 강원도 상원사
				\item [일시]	7월 16일 (목) 오전 7시 출발
				\item [금액]	
			\end{itemize}
			\end{block}						
		\end{frame}					

	%	---------------------------------------------------------- 백중기도 
	%		Frame
	%	----------------------------------------------------------
		\section{백중기도 및 영가천도 49재}
		\begin{frame} [t,plain]
		\frametitle{백중기도 및 영가천도 49재 }
			\begin{block} {백중기도 및 영가천도 49재 }

			\setlength{\leftmargini}{5em}			
			\begin{itemize}
				\item [입재]	 강원도 상원사
				\item [회향]	7월 16일 (목) 오전 7시 출발
				\item [망축기도]	영가1위 1만원
				\item [생축기도]	가족당 3만원
			\end{itemize}

			\end{block}						
		\end{frame}					




% ------------------------------------------------------------------------------
% End document
% ------------------------------------------------------------------------------





\end{document}


