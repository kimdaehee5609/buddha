%	-------------------------------------------------------------------------------
% 
%
%
%	-------------------------------------------------------------------------------

	\documentclass[12pt, a4paper, oneside]{book}
%	\documentclass[12pt, a4paper, landscape, oneside]{book}

		% --------------------------------- 페이지 스타일 지정
		\usepackage{geometry}
%		\geometry{landscape=true	}
		\geometry{top 			=10em}
		\geometry{bottom		=10em}
		\geometry{left			=8em}
		\geometry{right			=8em}
		\geometry{headheight	=4em} % 머리말 설치 높이
		\geometry{headsep		=2em} % 머리말의 본문과의 띠우기 크기
		\geometry{footskip		=4em} % 꼬리말의 본문과의 띠우기 크기
% 		\geometry{showframe}
	
%		paperwidth 	= left + width + right (1)
%		paperheight 	= top + height + bottom (2)
%		width 		= textwidth (+ marginparsep + marginparwidth) (3)
%		height 		= textheight (+ headheight + headsep + footskip) (4)



		%	===================================================================
		%	package
		%	===================================================================
%			\usepackage[hangul]{kotex}				% 한글 사용
			\usepackage{kotex}					% 한글 사용
			\usepackage[unicode]{hyperref}			% 한글 하이퍼링크 사용

		% ------------------------------ 수학 수식
			\usepackage{amssymb,amsfonts,amsmath}	% 수학 수식 사용
			\usepackage{mathtools}				% amsmath 확장판

			\usepackage{scrextend}				% 
		

		% ------------------------------ LIST
			\usepackage{enumerate}			%
			\usepackage{enumitem}			%
			\usepackage{tablists}				%	수학문제의 보기 등을 표현하는데 사용
										%	tabenum


		% ------------------------------ table 
			\usepackage{longtable}			%
			\usepackage{tabularx}			%
			\usepackage{tabu}				%




		% ------------------------------ 
			\usepackage{setspace}			%
			\usepackage{booktabs}		% table
			\usepackage{color}			%
			\usepackage{multirow}			%
			\usepackage{boxedminipage}	% 미니 페이지
			\usepackage[pdftex]{graphicx}	% 그림 사용
			\usepackage[final]{pdfpages}		% pdf 사용
			\usepackage{framed}			% pdf 사용

			
			\usepackage{fix-cm}	
			\usepackage[english]{babel}
	
		%	=======================================================================================
		% 	tikz package
		% 	
		% 	--------------------------------- 	
			\usepackage{tikz}%
			\usetikzlibrary{arrows,positioning,shapes}
			\usetikzlibrary{mindmap}			
			

		% --------------------------------- 	page
			\usepackage{afterpage}		% 다음페이지가 나온면 어떻게 하라는 명령 정의 패키지
%			\usepackage{fullpage}			% 잘못 사용하면 다 흐트러짐 주의해서 사용
%			\usepackage{pdflscape}		% 
			\usepackage{lscape}			%	 


			\usepackage{blindtext}
	
		% --------------------------------- font 사용
			\usepackage{pifont}				%
			\usepackage{textcomp}
			\usepackage{gensymb}
			\usepackage{marvosym}



		% Package --------------------------------- 

			\usepackage{tablists}				%


		% Package --------------------------------- 
			\usepackage[framemethod=TikZ]{mdframed}				% md framed package
			\usepackage{smartdiagram}								% smart diagram package



		% Package ---------------------------------    연습문제 

			\usepackage{exsheets}				%

			\SetupExSheets{solution/print=true}
			\SetupExSheets{question/type=exam}
			\SetupExSheets[points]{name=point,name-plural=points}


		% --------------------------------- 페이지 스타일 지정

		\usepackage[Sonny]		{fncychap}

			\makeatletter
			\ChNameVar	{\Large\bf}
			\ChNumVar	{\Huge\bf}
			\ChTitleVar		{\Large\bf}
			\ChRuleWidth	{0.5pt}
			\makeatother

%		\usepackage[Lenny]		{fncychap}
%		\usepackage[Glenn]		{fncychap}
%		\usepackage[Conny]		{fncychap}
%		\usepackage[Rejne]		{fncychap}
%		\usepackage[Bjarne]	{fncychap}
%		\usepackage[Bjornstrup]{fncychap}

		\usepackage{fancyhdr}
		\pagestyle{fancy}
		\fancyhead{} % clear all fields
		\fancyhead[LO]{\footnotesize \leftmark}
		\fancyhead[RE]{\footnotesize \leftmark}
		\fancyfoot{} % clear all fields
		\fancyfoot[LE,RO]{\large \thepage}
		%\fancyfoot[CO,CE]{\empty}
		\renewcommand{\headrulewidth}{1.0pt}
		\renewcommand{\footrulewidth}{0.4pt}
	
	
	
		%	--------------------------------------------------------------------------------------- 
		% 	tritlesec package
		% 	
		% 	
		% 	------------------------------------------------------------------ section 스타일 지정
	
			\usepackage{titlesec}
		
		% 	----------------------------------------------------------------- section 글자 모양 설정
			\titleformat*{\section}					{\large\bfseries}
			\titleformat*{\subsection}					{\normalsize\bfseries}
			\titleformat*{\subsubsection}				{\normalsize\bfseries}
%			\titleformat*{\paragraph}					{\normalsize\bfseries}
%			\titleformat*{\paragraph}					{\LARGE\bfseries}
%			\titleformat*{\paragraph}					{\Large\bfseries}
			\titleformat*{\paragraph}					{\large\bfseries}
			\titleformat*{\subparagraph}				{\normalsize\bfseries}
	
		% 	----------------------------------------------------------------- section 번호 설정
			\renewcommand{\thepart}				{\arabic{part}.}
			\renewcommand{\thesection}				{\arabic{section}.}
			\renewcommand{\thesubsection}			{\thesection\arabic{subsection}.}
			\renewcommand{\thesubsubsection}		{\thesubsection\arabic{subsubsection}}
			\renewcommand\theparagraph 			{$\blacksquare$ \hspace{3pt}}

		% 	----------------------------------------------------------------- section 페이지 나누기 설정
			\let\stdsection\section
			\renewcommand\section{\newpage\stdsection}



		%	--------------------------------------------------------------------------------------- 
		% 	\titlespacing*{commandi} {left} {before-sep} {after-sep} [right-sep]		
		% 	left
		%	before-sep		:  수직 전 간격
		% 	after-sep	 	:  수직으로 후 간격
		%	right-sep

			\titlespacing*{\section} 			{0pt}{1.0em}{1.0em}
			\titlespacing*{\subsection}	  		{0ex}{1.0em}{1.0em}
			\titlespacing*{\subsubsection}		{0ex}{1.0em}{1.0em}
			\titlespacing*{\paragraph}			{0em}{1.5em}{1.0em}
			\titlespacing*{\subparagraph}		{4em}{1.0em}{1.0em}
	
		%	\titlespacing*{\section} 			{0pt}{0.0\baselineskip}{0.0\baselineskip}
		%	\titlespacing*{\subsection}	  		{0ex}{0.0\baselineskip}{0.0\baselineskip}
		%	\titlespacing*{\subsubsection}		{6ex}{0.0\baselineskip}{0.0\baselineskip}
		%	\titlespacing*{\paragraph}			{6pt}{0.0\baselineskip}{0.0\baselineskip}
	

		% --------------------------------- recommend		섹션별 페이지 상단 여백
		\newcommand{\SectionMargin}				{\newpage  \null \vskip 2cm}
		\newcommand{\SubSectionMargin}			{\newpage  \null \vskip 2cm}
		\newcommand{\SubSubSectionMargin}		{\newpage  \null \vskip 2cm}


		%	--------------------------------------------------------------------------------------- 
		% 	toc 설정  - table of contents
		% 	
		% 	
		% 	----------------------------------------------------------------  문서 기본 사항 설정
			\setcounter{secnumdepth}{4} 		% 문단 번호 깊이
			\setcounter{tocdepth}{2} 			% 문단 번호 깊이 - 목차 출력시 출력 범위

			\setlength{\parindent}{0cm} 		% 문서 들여 쓰기를 하지 않는다.


		%	--------------------------------------------------------------------------------------- 
		% 	mini toc 설정
		% 	
		% 	
		% 	--------------------------------------------------------- 장의 목차  minitoc package
			\usepackage{minitoc}

			\setcounter{minitocdepth}{1}    	%  Show until subsubsections in minitoc
%			\setlength{\mtcindent}{12pt} 	% default 24pt
			\setlength{\mtcindent}{24pt} 	% default 24pt

		% 	--------------------------------------------------------- part toc
		%	\setcounter{parttocdepth}{2} 	%  default
			\setcounter{parttocdepth}{0}
		%	\setlength{\ptcindent}{0em}		%  default  목차 내용 들여 쓰기
			\setlength{\ptcindent}{0em}         


		% 	--------------------------------------------------------- section toc

			\renewcommand{\ptcfont}{\normalsize\rm} 		%  default
			\renewcommand{\ptcCfont}{\normalsize\bf} 	%  default
			\renewcommand{\ptcSfont}{\normalsize\rm} 	%  default


		%	=======================================================================================
		% 	tocloft package
		% 	
		% 	------------------------------------------ 목차의 목차 번호와 목차 사이의 간격 조정
			\usepackage{tocloft}

		% 	------------------------------------------ 목차의 내어쓰기 즉 왼쪽 마진 설정
			\setlength{\cftsecindent}{2em}			%  section

		% 	------------------------------------------ 목차의 목차 번호와 목차 사이의 간격 조정
			\setlength{\cftsecnumwidth}{2em}		%  section





		%	=======================================================================================
		% 	flowchart  package
		% 	
		% 	------------------------------------------ 목차의 목차 번호와 목차 사이의 간격 조정
			\usepackage{flowchart}
			\usetikzlibrary{arrows}


		%	=======================================================================================
		% 		makeindex package
		% 	
		% 	------------------------------------------ 목차의 목차 번호와 목차 사이의 간격 조정
%			\usepackage{makeindex}
%			\usepackage{makeidy}


		%	=======================================================================================
		% 		각주와 미주
		% 	

		\usepackage{endnotes} %미주 사용


		%	=======================================================================================
		% 	줄 간격 설정
		% 	
		% 	
		% 	--------------------------------- 	줄간격 설정
			\doublespace
%			\onehalfspace
%			\singlespace
		
		

	% 	============================================================================== itemi Global setting

	
		%	-------------------------------------------------------------------------------
		%		Vertical spacing
		%	-------------------------------------------------------------------------------
			\setlist[itemize]{topsep=0.0em}			% 상단의 여유치
			\setlist[itemize]{partopsep=0.0em}			% 
			\setlist[itemize]{parsep=0.0em}			% 
%			\setlist[itemize]{itemsep=0.0em}			% 
			\setlist[itemize]{noitemsep}				% 
			
		%	-------------------------------------------------------------------------------
		%		Horizontal spacing
		%	-------------------------------------------------------------------------------
			\setlist[itemize]{labelwidth=1em}			%  라벨의 표시 폭
			\setlist[itemize]{leftmargin=8em}			%  본문 까지의 왼쪽 여백  - 4em
			\setlist[itemize]{labelsep=3em} 			%  본문에서 라벨까지의 거리 -  3em
			\setlist[itemize]{rightmargin=0em}			% 오른쪽 여백  - 4em
			\setlist[itemize]{itemindent=0em} 			% 점 내민 거리 label sep 과 같은면 점위치 까지 내민다
			\setlist[itemize]{listparindent=3em}		% 본문 드려쓰기 간격
	
	
			\setlist[itemize]{ topsep=0.0em, 			%  상단의 여유치
						partopsep=0.0em, 		%  
						parsep=0.0em, 
						itemsep=0.0em, 
						labelwidth=1em, 
						leftmargin=2.5em,
						labelsep=2em,			%  본문에서 라벨 까지의 거리
						rightmargin=0em,		% 오른쪽 여백  - 4em
						itemindent=0em, 		% 점 내민 거리 label sep 과 같은면 점위치 까지 내민다
						listparindent=0em}		% 본문 드려쓰기 간격
	
%			\begin{itemize}
	
		%	-------------------------------------------------------------------------------
		%		Label
		%	-------------------------------------------------------------------------------
			\renewcommand{\labelitemi}{$\bullet$}
			\renewcommand{\labelitemii}{$\bullet$}
%			\renewcommand{\labelitemii}{$\cdot$}
			\renewcommand{\labelitemiii}{$\diamond$}
			\renewcommand{\labelitemiv}{$\ast$}		
	
%			\renewcommand{\labelitemi}{$\blacksquare$}   	% 사각형 - 찬것
%			\renewcommand\labelitemii{$\square$}		% 사각형 - 빈것	
			






% ------------------------------------------------------------------------------
% Begin document (Content goes below)
% ------------------------------------------------------------------------------
	\begin{document}
	
			\dominitoc
			\doparttoc			




			\title{불교 사찰}
			\author{김대희}
			\date{2020년 6월}
			\maketitle


			\tableofcontents 		% 목차 출력
%			\listoffigures 			% 그림 목차 출력
			\cleardoublepage
			\listoftables 			% 표 목차 출력





		\mdfdefinestyle	{con_specification} {
						outerlinewidth		=1pt			,%
						innerlinewidth		=2pt			,%
						outerlinecolor		=blue!70!black	,%
						innerlinecolor		=white 			,%
						roundcorner			=4pt			,%
						skipabove			=1em 			,%
						skipbelow			=1em 			,%
						leftmargin			=0em			,%
						rightmargin			=0em			,%
						innertopmargin		=2em 			,%
						innerbottommargin 	=2em 			,%
						innerleftmargin		=1em 			,%
						innerrightmargin		=1em 			,%
						backgroundcolor		=gray!4			,%
						frametitlerule		=true 			,%
						frametitlerulecolor	=white			,%
						frametitlebackgroundcolor=black		,%
						frametitleaboveskip=1em 			,%
						frametitlebelowskip=1em 			,%
						frametitlefontcolor=white 			,%
						}



%	================================================================== Part			불교이론
	\addtocontents{toc}{\protect\newpage}
	\part{불교 사찰}
	\noptcrule
	\parttoc				

%	-------------------------------------------------------------------- chapter 
	\chapter{2020년 경자년}


	\section{2020년 경자년}

			\begin{itemize}	[
						topsep=0.0em, 
						parsep=0.0em, 
						itemsep=0em, 
						leftmargin=6.0em, 
						labelwidth=3em, 
						labelsep=3em
						] 
			\item 01월 02일 목 성도재일
			\item 01월 02일 목 약사재일
			\item 01월 12일 일 지장재일
			\item 01월 18일 토 관음재일
			\item 01월 25일 토 초하루

			\item 02월 01일 토 약사재일
			\item 02월 08일 토 동안거 해제
			\item 02월 11일 화 지장재일
			\item 02월 17일 월 관음재일
			\item 02월 24일 월 초하루

			\item 03월 02일 월 출가재일
			\item 03월 02일 월 약사재일
			\item 03월 09일 월 열반재일
			\item 03월 12일 목 지장재일
			\item 03월 18일 수 관음재일
			\item 03월 24일 화 초하루
			\item 03월 31일 화 약사재일


			\item 04월 10일 금 지장재일
			\item 04월 16일 목 관음재일
			\item 04월 23일 목 초하루
			\item 04월 30일 목 약사재일

			\item 05월 07일 목 하안거결제

			\item 05월 10일 일 지장재일
			\item 05월 16일 토 관음재일
			\item 05월 23일 토 초하루
			\item 05월 30일 토 약사재일

			\item 06월 09일 화 지장재일
			\item 06월 15일 월 관음재일
			\item 06월 21일 일 초하루
			\item 06월 28일 일 약사재일

			\item 07월 08일 수 지장재일
			\item 07월 14일 화 관음재일
			\item 07월 21일 화 초하루
			\item 07월 28일 화 약사재일

			\item 08월 07일 금 지장재일
			\item 08월 13일 목 관음재일
			\item 08월 19일 수 초하루
			\item 08월 26일 수 약사재일


			\item 09월 02일 수 하안거 해제
			\item 09월 02일 수 백중(우란분절)


			\item 09월 05일 토 지장재일
			\item 09월 11일 금 관음재일
			\item 09월 17일 목 초하루
			\item 09월 24일 목 약사재일

			\item 10월 04일 일 지장재일
			\item 10월 10일 토 관음재일
			\item 10월 17일 토 초하루
			\item 10월 24일 토 약사재일

			\item 11월 03일 화 지장재일
			\item 11월 09일 월 관음재일
			\item 11월 15일 일 초하루
			\item 11월 22일 일 약사재일

			\item 11월 29일 일 동안거결제


			\item 12월 02일 수 지장재일
			\item 12월 08일 화 관음재일
			\item 12월 15일 화 초하루
			\item 12월 22일 화 약사재일

			\end{itemize}




%	================================================================== Part			불교이론
	\addtocontents{toc}{\protect\newpage}
	\part{불교 사찰 - 부산시}
	\noptcrule
	\parttoc				


%	-------------------------------------------------------------------- chapter 
	\chapter{부산시}


%	-------------------------------------------------------------------- chapter 
	\chapter{범어사}
	\minitoc


		\section{연혁}

		\section{위치}

		\section{구성}

		\section{전각}

		\section{보물}

		\section{행사}

%	-------------------------------------------------------------------- chapter 
	\chapter{홍범사}




%	-------------------------------------------------------------------- chapter 
	\chapter{여래선원}



%	-------------------------------------------------------------------- chapter 
	\chapter{용두산 미타선원}



	\chapter{대각사}

%	-------------------------------------------------------------------- chapter 
	\chapter{초량 소림사}









%	================================================================== Part			불교이론
	\addtocontents{toc}{\protect\newpage}
	\part{불교 사찰 - 경상남도}
	\noptcrule
	\parttoc				


%	-------------------------------------------------------------------- chapter 
	\chapter{경상남도}



%	-------------------------------------------------------------------- chapter 
	\chapter{양산 통도사}


%	-------------------------------------------------------------------- chapter 
	\chapter{내원사}
	\minitoc


		\section{연혁}

		\section{위치}

		\section{구성}

		\section{전각}

		\section{보물}

		\section{행사}


		\section{익선암}


%	-------------------------------------------------------------------- chapter 
	\chapter{여여선원 }


%	-------------------------------------------------------------------- chapter 
	\chapter{만어사}


%	-------------------------------------------------------------------- chapter 
	\chapter{표충사}



%	================================================================== Part			경상북도
	\addtocontents{toc}{\protect\newpage}
	\part{불교 사찰 - 경상북도}
	\noptcrule
	\parttoc				


%	-------------------------------------------------------------------- chapter 경상북도
	\chapter{ 경상북도 }


%	-------------------------------------------------------------------- chapter 경주시
	\chapter{ 경주시 }


%	================================================================== Part			전라남도
	\addtocontents{toc}{\protect\newpage}
	\part{불교 사찰 - 전라남도}
	\noptcrule
	\parttoc				


%	-------------------------------------------------------------------- chapter 경상북도
	\chapter{ 전라남도 }


%	-------------------------------------------------------------------- chapter 순천시 송광사
	\chapter{ 순천시 송광사 }


%	-------------------------------------------------------------------- chapter 순천시 선암사
	\chapter{ 순천시 선암사 }



%	================================================================== Part			불교이론
	\addtocontents{toc}{\protect\newpage}
	\part{불교 사찰 - 전라북도}
	\noptcrule
	\parttoc				


%	-------------------------------------------------------------------- chapter 
	\chapter{전라북도}

%	-------------------------------------------------------------------- chapter 
	\chapter{귀신사}


%	-------------------------------------------------------------------- chapter 
	\chapter{금산사}
	\minitoc


		\section{연혁}

		\section{위치}

		\section{구성}

		\section{전각}

		\section{보물}


%	================================================================== Part		강원도
	\addtocontents{toc}{\protect\newpage}
	\part{ 강원도 }
	\noptcrule
	\parttoc				


%	-------------------------------------------------------------------- chapter 
	\chapter{ 강원도 }




%	================================================================== Part	경기도
	\addtocontents{toc}{\protect\newpage}
	\part{ 경기도 }
	\noptcrule
	\parttoc				


%	-------------------------------------------------------------------- chapter 
	\chapter{ 경기도 }



%	================================================================== Part	서울시
	\addtocontents{toc}{\protect\newpage}
	\part{ 서울시 }
	\noptcrule
	\parttoc				


%	-------------------------------------------------------------------- chapter 
	\chapter{ 서울시 }









% ------------------------------------------------------------------------------
% End document
% ------------------------------------------------------------------------------
\end{document}





