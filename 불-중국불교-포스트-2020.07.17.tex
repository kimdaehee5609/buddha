%	------------------------------------------------------------------------------
%
%		작성 : 2020년 7월 18일 첫 작업
%
%

%	\documentclass[25pt, a1paper]{tikzposter}
%	\documentclass[25pt, a0paper, landscape]{tikzposter}
	\documentclass[25pt, a1paper ]{tikzposter}

%	\documentclass[25pt, a1paper]{tikzposter}
%	\documentclass[25pt, a1paper]{tikzposter}
%	\documentclass[25pt, a1paper]{tikzposter}

% 	12pt  14pt 17pt  20pt  25pt
%
%	a0 a1 a2
%
%	landscape  portrait
%

	%% Tikzposter is highly customizable: please see
	%% https://bitbucket.org/surmann/tikzposter/downloads/styleguide.pdf

	%	========================================================== 	Package
		\usepackage{kotex}						% 한글 사용


%% Available themes: see also
%% https://bitbucket.org/surmann/tikzposter/downloads/themes.pdf
%	\usetheme{Default}
%	\usetheme{Rays}
%	\usetheme{Basic}
	\usetheme{Simple}
%	\usetheme{Envelope}
%	\usetheme{Wave}
%	\usetheme{Board}
%	\usetheme{Autumn}
%	\usetheme{Desert}

%% Further changes to the title etc is possible
%	\usetitlestyle{Default}			%
%	\usetitlestyle{Basic}				%
%	\usetitlestyle{Empty}				%
%	\usetitlestyle{Filled}				%
%	\usetitlestyle{Envelope}			%
%	\usetitlestyle{Wave}				%
%	\usetitlestyle{verticalShading}	%


%	\usebackgroundstyle{Default}
%	\usebackgroundstyle{Rays}
%	\usebackgroundstyle{VerticalGradation}
%	\usebackgroundstyle{BottomVerticalGradation}
%	\usebackgroundstyle{Empty}

%	\useblockstyle{Default}
%	\useblockstyle{Basic}
%	\useblockstyle{Minimal}		% 이것은 간단함
%	\useblockstyle{Envelope}		% 
%	\useblockstyle{Corner}		% 사각형
%	\useblockstyle{Slide}			%	띠모양  
	\useblockstyle{TornOut}		% 손그림모양


	\usenotestyle{Default}
%	\usenotestyle{Corner}
%	\usenotestyle{VerticalShading}
%	\usenotestyle{Sticky}

%	\usepackage{fontspec}
%	\setmainfont{FreeSerif}
%	\setsansfont{FreeSans}

%	------------------------------------------------------------------------------ 제목

	\title{ 중국 불교 }

	\author{ 2020년 7월  18일 }

%	\institute{서영엔지니어링}
%	\titlegraphic{\includegraphics[width=7cm]{IMG_1934}}

	%% Optional title graphic
	%\titlegraphic{\includegraphics[width=7cm]{IMG_1934}}
	%% Uncomment to switch off tikzposter footer
	% \tikzposterlatexaffectionproofoff

\begin{document}

	\maketitle



	\begin{columns}

		\column{0.5}

%	------------------------------------------------------------------------------ 왕조
			\block{■  왕조 }
			{
					\setlength{\leftmargini}{2em}
					\setlength{\labelsep} {1em}
					\begin{itemize}
					\item 하 최초
					\item 은
					\item 주
					\item 춘추5패 / 전국7
					\item 진 (BC221 - 206 )
					\item 한 (202-220)
					\item 삼국시대  위(조조) 촉(유비) 오(손권)
					\item 위

					\item 진 265-420
					\item 서진 265-316
					\item 동진 317-420

					\item 남북조 5유목민족(북위,북제,북부) 한족(송,제,양,진)
					\item 수 581-619
					\item 당 618-907
					\item 오대 십국 시대

					\item 송
					\item 원
					\item 명
					\item 청 여진족(만주족)
					\item 중화민국
					\item 중화 인민 공화국
					\end{itemize}
			}


%	------------------------------------------------------------------------------ 전래
			\block{■  전래 }
			{
					\setlength{\leftmargini}{2em}
					\setlength{\labelsep} {1em}
					\begin{itemize}
					\item 1. 프라남아사나(기도하는 자세) 
					\item 2. 하스타 우딴아사나( 팔을 들어올린 자세)
					\item 3. 파다하스타사나(손을 발 쪽으로 향하놓는 자세)
					\item 4. 아슈와 산찰라나사나(말에 올라탄 자세)
					\item 5. 파르바타사나(산 자세)
					\item 6. 아슈탕가 나마스카라(여덟 부분으로 경배하는 자세)
					\item 7. 부장가사나(코브라 자세)
					\item 8. 파르바타사나(산 자세)
					\item 9. 아슈와 산찰라나사나( 말에 올라탄 자세)
					\item 10. 파다하스타사나(손을 발 쪽으로 향하는 자세)
					\item 11. 하스타 우딴아사나(팔을 들어올린 자세)
					\item 12. 프라남아사나(기도하는 자세)
					\end{itemize}
			}

%	------------------------------------------------------------------------------ 종파
			\block{■  종파 }
			{
					\setlength{\leftmargini}{3em}
					\setlength{\labelsep} {1em}
				\begin{LARGE}
					\begin{itemize}
					\item 
					\item 
					\item 
					\item 
					\end{itemize}
				\end{LARGE}
			}



%	------------------------------------------------------------------------------
		\block{■ 정토종 }
		{
			정토교는 아미타불의 구원을 믿고, 
				염불을 외어 서방극락정토(西方極樂淨土)에 왕생(往生)하여 깨달음을 얻는다고 설하는 종파이다.
		}		



	%	====== ====== ====== ====== ====== 
		\column{0.5}

%	------------------------------------------------------------------------------
			\block [titleleft,linewidth=3mm]{■ 물품 }
			{				
			\setlength{\leftmargini}{5em}			
			\setlength{\labelsep}{1em} % horizontal space from bullet to text (as needed)
			\begin{LARGE}
			\begin{itemize}
			\item [바지]
			\item [상의]
			\item [매트]
			\item [블럭]
			\end{itemize}
			\end{LARGE}
		}


%	------------------------------------------------------------------------------ 잡지
			\block [titleleft,linewidth=3mm]{■ 잡지 }
			{				
			\setlength{\leftmargini}{3em}			
			\setlength{\labelsep}{1em} % horizontal space from bullet to text (as needed)
			\begin{LARGE}
			\begin{itemize}
			\item 요가 저널
			\item 
			\end{itemize}
			\end{LARGE}
		}


%	------------------------------------------------------------------------------ 쇼핑몰
			\block [titleleft,linewidth=3mm]{■ 쇼핑몰 }
			{				
			\setlength{\leftmargini}{3em}			
			\setlength{\labelsep}{1em} % horizontal space from bullet to text (as needed)
			\begin{LARGE}
			\begin{itemize}
			\item manduka 만두카
			\item avocado 아보카드코리아
			\item meditation 메디테이
			\item AUMNIE
			\item almondon 아몬드온
			\item SKULLPIG 스컬피그

			\end{itemize}
			\end{LARGE}
		}


	\end{columns}




\end{document}


		\begin{huge}
		\end{huge}

		\begin{LARGE}
		\end{LARGE}

		\begin{Large}
		\end{Large}

		\begin{large}
		\end{large}

