%	------------------------------------------------------------------------------
%
%	작성 : 2020년 7월 22일 첫 작업
%	
%	천수경에서 떼어내서 따로 한페이로 편집	
%    

%	\documentclass[25pt, a1paper]{tikzposter}
%	\documentclass[25pt, a0paper, landscape]{tikzposter}
	\documentclass[17pt, a2paper ]{tikzposter}

%	\documentclass[25pt, a1paper]{tikzposter}
%	\documentclass[25pt, a1paper]{tikzposter}
%	\documentclass[25pt, a1paper]{tikzposter}

% 	12pt  14pt 17pt  20pt  25pt
%
%	a0 a1 a2
%
%	landscape  portrait
%

	%% Tikzposter is highly customizable: please see
	%% https://bitbucket.org/surmann/tikzposter/downloads/styleguide.pdf

	%	========================================================== 	Package
		\usepackage{kotex}						% 한글 사용


%% Available themes: see also
%% https://bitbucket.org/surmann/tikzposter/downloads/themes.pdf
%	\usetheme{Default}
%	\usetheme{Rays}
%	\usetheme{Basic}
	\usetheme{Simple}
%	\usetheme{Envelope}
%	\usetheme{Wave}
%	\usetheme{Board}
%	\usetheme{Autumn}
%	\usetheme{Desert}

%% Further changes to the title etc is possible
%	\usetitlestyle{Default}			%
%	\usetitlestyle{Basic}				%
%	\usetitlestyle{Empty}				%
%	\usetitlestyle{Filled}				%
%	\usetitlestyle{Envelope}			%
%	\usetitlestyle{Wave}				%
%	\usetitlestyle{verticalShading}	%


%	\usebackgroundstyle{Default}
%	\usebackgroundstyle{Rays}
%	\usebackgroundstyle{VerticalGradation}
%	\usebackgroundstyle{BottomVerticalGradation}
%	\usebackgroundstyle{Empty}

%	\useblockstyle{Default}
%	\useblockstyle{Basic}
%	\useblockstyle{Minimal}		% 이것은 간단함
%	\useblockstyle{Envelope}		% 
%	\useblockstyle{Corner}		% 사각형
%	\useblockstyle{Slide}			%	띠모양  
	\useblockstyle{TornOut}		% 손그림모양


	\usenotestyle{Default}
%	\usenotestyle{Corner}
%	\usenotestyle{VerticalShading}
%	\usenotestyle{Sticky}

%	\usepackage{fontspec}
%	\setmainfont{FreeSerif}
%	\setsansfont{FreeSans}

%	------------------------------------------------------------------------------ 제목

	\title{ 摩訶般若波羅蜜多心經  마하 반야 바라밀다 심경 }

	\author{ 2020년 7월 }

%	\institute{김대희}
%	\titlegraphic{\includegraphics[width=7cm]{IMG_1934}}

	%% Optional title graphic
	%\titlegraphic{\includegraphics[width=7cm]{IMG_1934}}
	%% Uncomment to switch off tikzposter footer
	% \tikzposterlatexaffectionproofoff

\begin{document}

	\maketitle





%	------------------------------------------------------------------------------ 神妙章句大陀羅尼 신묘장구대다라니
			\block{■  摩訶般若波羅蜜多心經  마하 반야 바라밀다 심경  }
			{
				\begin{LARGE}



觀自在菩薩 行深般若波羅蜜多時 照見 五蘊皆空 度 一切苦厄 \\
관자재보살 행심반야바라밀다시 조견 오온개공 도 일체고액 \\

舍利子 色不異空 空不異色 色卽是空 空卽是色 受想行識 亦復如是\\
사리자 색불이공 공불이색 색즉시공 공즉시색 수상행식 역부여시

舍利子 是諸法空相 不生不滅 不垢不淨 不增不減\\
사리자 시제법공상 불생불멸 불구부정 부증불감

是故 空中無色 無受想行識\\
시고 공중무색 무수상행식

無眼耳鼻舌身意 無色聲香味觸法 無眼界 乃至 無意識界\\
무안이비설신의 무색성향미촉법 무안계 내지 무의식계

無無明 亦無無明盡 乃至 無老死 亦無老死盡\\
무무명 역무무명진 내지 무노사 역무노사진

無苦集滅道 無智亦無得\\
무고집멸도 무지역무득

以無所得故 菩提薩埵 依般若波羅蜜多故\\
이무소득고 보리살타 의반야바라밀다고

心無罣礙 無罣礙故 無有恐怖 遠離顚倒夢想 究竟涅槃\\
심무가애 무가애고 무유공포 원리전도몽상 구경열반

三世諸佛 依般若波羅蜜多故 得阿耨多羅三藐三菩提\\
삼세제불 의반야바라밀다고 득아뇩다라삼막삼보리

故知 般若波羅蜜多 是大神呪 是大明呪 是無上呪 是無等等呪 能除 一切苦 眞實不虛\\
고지 반야바라밀다 시대신주 시대명주 시무상주 시무등등주 능제 일체고 진실불허

故說 般若波羅蜜多呪 卽說呪曰\\
고설 반야바라밀다주 즉설주왈 \\

揭諦揭諦 波羅揭諦 波羅僧揭諦 菩提娑婆訶\\
아제아제 바라아제 바라승아제 모지사바하

揭諦揭諦 波羅揭諦 波羅僧揭諦 菩提娑婆訶\\
아제아제 바라아제 바라승아제 모지사바하

揭諦揭諦 波羅揭諦 波羅僧揭諦 菩提娑婆訶\\
아제아제 바라아제 바라승아제 모지사바하

				\end{LARGE}
			}




\end{document}


		\begin{huge}
		\end{huge}

		\begin{LARGE}
		\end{LARGE}

		\begin{Large}
		\end{Large}

		\begin{large}
		\end{large}

