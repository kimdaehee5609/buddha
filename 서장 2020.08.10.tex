%	-------------------------------------------------------------------------------
% 
%		2020년 6월 16일 첫작업
%		2020년 6월 30일 장로 자각색선사 좌선의 추가
%		2020년 7월 13일 장로 자각 종색선사 좌선의 보완
%
%
%
%
%
%
%	-------------------------------------------------------------------------------

%	\documentclass[12pt, a3paper, oneside]{book}
	\documentclass[12pt, a4paper, oneside]{book}
%	\documentclass[12pt, a4paper, landscape, oneside]{book}

		% --------------------------------- 페이지 스타일 지정
		\usepackage{geometry}
%		\geometry{landscape=true	}
		\geometry{top 			=10em}
		\geometry{bottom			=10em}
		\geometry{left			=8em}
		\geometry{right			=8em}
		\geometry{headheight		=4em} % 머리말 설치 높이
		\geometry{headsep		=2em} % 머리말의 본문과의 띠우기 크기
		\geometry{footskip		=4em} % 꼬리말의 본문과의 띠우기 크기
% 		\geometry{showframe}
	
%		paperwidth 	= left + width + right (1)
%		paperheight 	= top + height + bottom (2)
%		width 		= textwidth (+ marginparsep + marginparwidth) (3)
%		height 		= textheight (+ headheight + headsep + footskip) (4)



		%	===================================================================
		%	package
		%	===================================================================
%			\usepackage[hangul]{kotex}				% 한글 사용
			\usepackage{kotex}					% 한글 사용
			\usepackage[unicode]{hyperref}			% 한글 하이퍼링크 사용

		% ------------------------------ 수학 수식
			\usepackage{amssymb,amsfonts,amsmath}	% 수학 수식 사용
			\usepackage{mathtools}				% amsmath 확장판

			\usepackage{scrextend}				% 
		

		% ------------------------------ LIST
			\usepackage{enumerate}			%
			\usepackage{enumitem}			%
			\usepackage{tablists}				%	수학문제의 보기 등을 표현하는데 사용
										%	tabenum


		% ------------------------------ table 
			\usepackage{longtable}			%
			\usepackage{tabularx}			%
			\usepackage{tabu}				%




		% ------------------------------ 
			\usepackage{setspace}			%
			\usepackage{booktabs}		% table
			\usepackage{color}			%
			\usepackage{multirow}			%
			\usepackage{boxedminipage}	% 미니 페이지
			\usepackage[pdftex]{graphicx}	% 그림 사용
			\usepackage[final]{pdfpages}		% pdf 사용
			\usepackage{framed}			% pdf 사용

			
			\usepackage{fix-cm}	
			\usepackage[english]{babel}
	
		%	=======================================================================================
		% 	tikz package
		% 	
		% 	--------------------------------- 	
			\usepackage{tikz}%
			\usetikzlibrary{arrows,positioning,shapes}
			\usetikzlibrary{mindmap}			
			

		% --------------------------------- 	page
			\usepackage{afterpage}		% 다음페이지가 나온면 어떻게 하라는 명령 정의 패키지
%			\usepackage{fullpage}			% 잘못 사용하면 다 흐트러짐 주의해서 사용
%			\usepackage{pdflscape}		% 
			\usepackage{lscape}			%	 


			\usepackage{blindtext}
	
		% --------------------------------- font 사용
			\usepackage{pifont}				%
			\usepackage{textcomp}
			\usepackage{gensymb}
			\usepackage{marvosym}



		% Package --------------------------------- 

			\usepackage{tablists}				%


		% Package --------------------------------- 
			\usepackage[framemethod=TikZ]{mdframed}				% md framed package
			\usepackage{smartdiagram}								% smart diagram package



		% Package ---------------------------------    연습문제 

			\usepackage{exsheets}				%

			\SetupExSheets{solution/print=true}
			\SetupExSheets{question/type=exam}
			\SetupExSheets[points]{name=point,name-plural=points}


		% --------------------------------- 페이지 스타일 지정

		\usepackage[Sonny]		{fncychap}

			\makeatletter
			\ChNameVar	{\Large\bf}
			\ChNumVar	{\Huge\bf}
			\ChTitleVar		{\Large\bf}
			\ChRuleWidth	{0.5pt}
			\makeatother

%		\usepackage[Lenny]		{fncychap}
%		\usepackage[Glenn]		{fncychap}
%		\usepackage[Conny]		{fncychap}
%		\usepackage[Rejne]		{fncychap}
%		\usepackage[Bjarne]	{fncychap}
%		\usepackage[Bjornstrup]{fncychap}

		\usepackage{fancyhdr}
		\pagestyle{fancy}
		\fancyhead{} % clear all fields
		\fancyhead[LO]{\footnotesize \leftmark}
		\fancyhead[RE]{\footnotesize \leftmark}
		\fancyfoot{} % clear all fields
		\fancyfoot[LE,RO]{\large \thepage}
		%\fancyfoot[CO,CE]{\empty}
		\renewcommand{\headrulewidth}{1.0pt}
		\renewcommand{\footrulewidth}{0.4pt}
	
	
	
		%	--------------------------------------------------------------------------------------- 
		% 	tritlesec package
		% 	
		% 	
		% 	------------------------------------------------------------------ section 스타일 지정
	
			\usepackage{titlesec}
		
		% 	----------------------------------------------------------------- section 글자 모양 설정
			\titleformat*{\section}					{\large\bfseries}
			\titleformat*{\subsection}				{\normalsize\bfseries}
			\titleformat*{\subsubsection}			{\normalsize\bfseries}
			\titleformat*{\paragraph}				{\normalsize\bfseries}
			\titleformat*{\subparagraph}				{\normalsize\bfseries}
	
		% 	----------------------------------------------------------------- section 번호 설정
			\renewcommand{\thepart}				{\arabic{part}.}
			\renewcommand{\thesection}				{\arabic{section}.}
			\renewcommand{\thesubsection}			{\thesection\arabic{subsection}.}
			\renewcommand{\thesubsubsection}		{\thesubsection\arabic{subsubsection}}
			\renewcommand\theparagraph 			{$\blacksquare$ \hspace{3pt}}

		% 	----------------------------------------------------------------- section 페이지 나누기 설정
			\let\stdsection\section
			\renewcommand\section{\newpage\stdsection}



		%	--------------------------------------------------------------------------------------- 
		% 	\titlespacing*{commandi} {left} {before-sep} {after-sep} [right-sep]		
		% 	left
		%	before-sep		:  수직 전 간격
		% 	after-sep	 	:  수직으로 후 간격
		%	right-sep

			\titlespacing*{\section} 			{0pt}{1.0em}{1.0em}
			\titlespacing*{\subsection}	  		{0ex}{1.0em}{1.0em}
			\titlespacing*{\subsubsection}		{0ex}{1.0em}{1.0em}
			\titlespacing*{\paragraph}			{0em}{1.5em}{1.0em}
			\titlespacing*{\subparagraph}		{4em}{1.0em}{1.0em}
	
		%	\titlespacing*{\section} 			{0pt}{0.0\baselineskip}{0.0\baselineskip}
		%	\titlespacing*{\subsection}	  		{0ex}{0.0\baselineskip}{0.0\baselineskip}
		%	\titlespacing*{\subsubsection}		{6ex}{0.0\baselineskip}{0.0\baselineskip}
		%	\titlespacing*{\paragraph}			{6pt}{0.0\baselineskip}{0.0\baselineskip}
	

		% --------------------------------- recommend		섹션별 페이지 상단 여백
		\newcommand{\SectionMargin}				{\newpage  \null \vskip 2cm}
		\newcommand{\SubSectionMargin}			{\newpage  \null \vskip 2cm}
		\newcommand{\SubSubSectionMargin}		{\newpage  \null \vskip 2cm}


		%	--------------------------------------------------------------------------------------- 
		% 	toc 설정  - table of contents
		% 	
		% 	
		% 	----------------------------------------------------------------  문서 기본 사항 설정
			\setcounter{secnumdepth}{4} 		% 문단 번호 깊이
			\setcounter{tocdepth}{2} 			% 문단 번호 깊이 - 목차 출력시 출력 범위

			\setlength{\parindent}{0cm} 		% 문서 들여 쓰기를 하지 않는다.


		%	--------------------------------------------------------------------------------------- 
		% 	mini toc 설정
		% 	
		% 	
		% 	--------------------------------------------------------- 장의 목차  minitoc package
			\usepackage{minitoc}

			\setcounter{minitocdepth}{1}    	%  Show until subsubsections in minitoc
%			\setlength{\mtcindent}{12pt} 	% default 24pt
			\setlength{\mtcindent}{24pt} 	% default 24pt

		% 	--------------------------------------------------------- part toc
		%	\setcounter{parttocdepth}{2} 	%  default
			\setcounter{parttocdepth}{0}
		%	\setlength{\ptcindent}{0em}		%  default  목차 내용 들여 쓰기
			\setlength{\ptcindent}{0em}         


		% 	--------------------------------------------------------- section toc

			\renewcommand{\ptcfont}{\normalsize\rm} 		%  default
			\renewcommand{\ptcCfont}{\normalsize\bf} 	%  default
			\renewcommand{\ptcSfont}{\normalsize\rm} 	%  default


		%	=======================================================================================
		% 	tocloft package
		% 	
		% 	------------------------------------------ 목차의 목차 번호와 목차 사이의 간격 조정
			\usepackage{tocloft}

		% 	------------------------------------------ 목차의 내어쓰기 즉 왼쪽 마진 설정
			\setlength{\cftsecindent}{2em}			%  section

		% 	------------------------------------------ 목차의 목차 번호와 목차 사이의 간격 조정
			\setlength{\cftsecnumwidth}{2em}		%  section





		%	=======================================================================================
		% 	flowchart  package
		% 	
		% 	------------------------------------------ 목차의 목차 번호와 목차 사이의 간격 조정
			\usepackage{flowchart}
			\usetikzlibrary{arrows}


		%	=======================================================================================
		% 		makeindex package
		% 	
		% 	------------------------------------------ 목차의 목차 번호와 목차 사이의 간격 조정
%			\usepackage{makeindex}
%			\usepackage{makeidy}


		%	=======================================================================================
		% 		각주와 미주
		% 	

		\usepackage{endnotes} %미주 사용


		%	=======================================================================================
		% 	줄 간격 설정
		% 	
		% 	
		% 	--------------------------------- 	줄간격 설정
			\doublespace
%			\onehalfspace
%			\singlespace
		
		

	% 	============================================================================== itemi Global setting

	
		%	-------------------------------------------------------------------------------
		%		Vertical spacing
		%	-------------------------------------------------------------------------------
			\setlist[itemize]{topsep=0.0em}			% 상단의 여유치
			\setlist[itemize]{partopsep=0.0em}			% 
			\setlist[itemize]{parsep=0.0em}			% 
%			\setlist[itemize]{itemsep=0.0em}			% 
			\setlist[itemize]{noitemsep}				% 
			
		%	-------------------------------------------------------------------------------
		%		Horizontal spacing
		%	-------------------------------------------------------------------------------
			\setlist[itemize]{labelwidth=1em}			%  라벨의 표시 폭
			\setlist[itemize]{leftmargin=8em}			%  본문 까지의 왼쪽 여백  - 4em
			\setlist[itemize]{labelsep=3em} 			%  본문에서 라벨까지의 거리 -  3em
			\setlist[itemize]{rightmargin=0em}			% 오른쪽 여백  - 4em
			\setlist[itemize]{itemindent=0em} 			% 점 내민 거리 label sep 과 같은면 점위치 까지 내민다
			\setlist[itemize]{listparindent=3em}		% 본문 드려쓰기 간격
	
	
			\setlist[itemize]{ topsep=0.0em, 			%  상단의 여유치
						partopsep=0.0em, 		%  
						parsep=0.0em, 
						itemsep=0.0em, 
						labelwidth=1em, 
						leftmargin=2.5em,
						labelsep=2em,			%  본문에서 라벨 까지의 거리
						rightmargin=0em,		% 오른쪽 여백  - 4em
						itemindent=0em, 		% 점 내민 거리 label sep 과 같은면 점위치 까지 내민다
						listparindent=0em}		% 본문 드려쓰기 간격
	
%			\begin{itemize}
	
		%	-------------------------------------------------------------------------------
		%		Label
		%	-------------------------------------------------------------------------------
			\renewcommand{\labelitemi}{$\bullet$}
			\renewcommand{\labelitemii}{$\bullet$}
%			\renewcommand{\labelitemii}{$\cdot$}
			\renewcommand{\labelitemiii}{$\diamond$}
			\renewcommand{\labelitemiv}{$\ast$}		
	
%			\renewcommand{\labelitemi}{$\blacksquare$}   	% 사각형 - 찬것
%			\renewcommand\labelitemii{$\square$}		% 사각형 - 빈것	
			






% ------------------------------------------------------------------------------
% Begin document (Content goes below)
% ------------------------------------------------------------------------------
	\begin{document}
	
			\dominitoc
			\doparttoc			




			\title{서장}
			\author{붓다선원 김대희}
			\date{2020년 6월}
			\maketitle


			\tableofcontents 		% 목차 출력
%			\listoffigures 			% 그림 목차 출력
			\cleardoublepage
			\listoftables 			% 표 목차 출력





		\mdfdefinestyle	{con_specification} {
						outerlinewidth		=1pt			,%
						innerlinewidth		=2pt			,%
						outerlinecolor		=blue!70!black	,%
						innerlinecolor		=white 			,%
						roundcorner			=4pt			,%
						skipabove			=1em 			,%
						skipbelow			=1em 			,%
						leftmargin			=0em			,%
						rightmargin			=0em			,%
						innertopmargin		=2em 			,%
						innerbottommargin 	=2em 			,%
						innerleftmargin		=1em 			,%
						innerrightmargin		=1em 			,%
						backgroundcolor		=gray!4			,%
						frametitlerule		=true 			,%
						frametitlerulecolor	=white			,%
						frametitlebackgroundcolor=black		,%
						frametitleaboveskip=1em 			,%
						frametitlebelowskip=1em 			,%
						frametitlefontcolor=white 			,%
						}



%	================================================================== Part			붓다 선원
	\addtocontents{toc}{\protect\newpage}
	\part{붓다선원}
	\noptcrule
	\parttoc				

%	================================================================== Part			선원
%	\addtocontents{toc}{\protect\newpage}
	\chapter{선원}
	\noptcrule

	\newpage	
	\minitoc


% ----------------------------------------------------------------------------- 선원장
%
% -----------------------------------------------------------------------------
	\section{선원장 도경 스님 }


\paragraph{}
최근 부산 대연동 문화회관 인근에 붓다선원을 개원한 도경 스님은 “교학과 수행을 함께 이어가는 도심 속 휴식과 충전의 도량이 될 것”이라고 포부를 밝혔다. 지난1월 부산 동명불원 주지 소임을 회향한 스님은 한 대학 인근의 강의실을 활용해 인연이 닿는 불자들과 공부 모임을 지속해 왔다. 이들이 보다 여법한 공간에서 불교 공부를 이어갈 수 있도록 돕기 위해 지난 2월 붓다선원 공간을 마련, 리모델링 과정을 거쳐 2013년 6월23일 탱화 점안식을 갖고 본격적인 운영을 시작했다.


\paragraph{}
특히 도경 스님은 “붓다선원은 불교대학 기초과정을 통해 부처님의 생애를 배우면서 부처님의 가르침에 근간을 둔 삶을 지도할 것”이라며 “부처님의 삶을 이해하고 나면 교학과 수행이 자연스럽게 동반된다. 시민선방 역시 불자들에게 항상 열어 놓고 선과 교를 함께 수행할 수 있도록 이끌 것”이라고 소개했다. 5월23일 개강한 불교대학의 경우 기초주간반이 30명, 야간반이 50명 넘게 수강 중이며 선방 역시 10여 명이 방부를 들여 정진 중이다.


\paragraph{}
붓다선원은 대지면적 140평, 지하1층 지상3층 규모의 단독 건물이다. 연면적 270평 규모에 법당과 선방, 요사채, 공양간을 갖추고 있다. 전체적으로 군더더기 없이 깔끔한 분위기가 산중 선방을 연상시킬 정도로 정갈하다. 특히 1층에 위치한 선방은 이중문과 복도, 탈의실, 지대방까지 갖추고 있어 수행 공간의 가치를 높였다.


부산=주영미 기자

출처 : 법보신문(http://www.beopbo.com)





% ----------------------------------------------------------------------------- 직원
%
% -----------------------------------------------------------------------------
	\section{직원}


\begin{itemize}[					
		topsep=0.0em,			
		parsep=0.0em,			
		itemsep=0em,			
		leftmargin=		2	em,
		labelwidth=1em,			
		labelsep=1em			
]					
	\item		\textbf{정구업진언} 
	\item	원주실 보살
	\item	법당 보살
	\item	공양간 보살
\end{itemize}					


% ----------------------------------------------------------------------------- 교육 프로그램
%
% -----------------------------------------------------------------------------
	\section{교육 프로그램}



% ----------------------------------------------------------------------------- 신도회
%
% -----------------------------------------------------------------------------
	\section{신도회}







%	================================================================== Part			불자운동
%	\addtocontents{toc}{\protect\newpage}
	\chapter{ 불자운동 }
	\noptcrule

	\newpage	
	\minitoc


% ----------------------------------------------------------------------------- 선원4대 불자운동
%
% -----------------------------------------------------------------------------
	\section{붓다선원 4대 불자운동 }


\begin{itemize}[					
		topsep=0.0em,			
		parsep=0.0em,			
		itemsep=0em,			
		leftmargin=		2	em,
		labelwidth=1em,			
		labelsep=1em			
]					
	\item	공부하는 불자가 되겠읍니다
	\item	수행하는 불자가 되겠읍니다
	\item	실천하는 불자가 되겠읍니다
	\item	포교하는 불자가 되겠읍니다
\end{itemize}					




% ----------------------------------------------------------------------------- 선원 3대 구경운동
%
% -----------------------------------------------------------------------------
	\section{붓다선원 3대 구경운동 }


\begin{itemize}[					
		topsep=0.0em,			
		parsep=0.0em,			
		itemsep=0em,			
		leftmargin=		2	em,
		labelwidth=1em,			
		labelsep=1em			
]					
	\item	맑은 마음 - 마음을 향기롭게
	\item	밝은 사회 - 사회를 향기롭게
	\item	좋은 세상 - 세상을 향기롭게
\end{itemize}					


% ----------------------------------------------------------------------------- 함께하는 세상
%
% -----------------------------------------------------------------------------
	\section{함께하는 세상 }


\begin{itemize}[					
		topsep=0.0em,			
		parsep=0.0em,			
		itemsep=0em,			
		leftmargin=		2	em,
		labelwidth=1em,			
		labelsep=1em			
]					
	\item	고맙습니다
	\item	감사합니다
	\item	사랑합니다
\end{itemize}					




%	================================================================== Part			장로 자각 종색 선사 좌선의
%	\addtocontents{toc}{\protect\newpage}
	\chapter{장로 자각 종색 선사 좌선의}
	\noptcrule

	\newpage	
	\minitoc

% ----------------------------------------------------------------------------- 장로 자각 종색 선사
%
% -----------------------------------------------------------------------------
	\section{장로 자각 종색 선사 長蘆 慈覺 宗賾 禪師 }


%	https://m.blog.naver.com/bhjang3/220606570356

\paragraph{}
	좌선의의 작자인 종색선사는 유명한 선원청규의 편자이기도 한데, 그이 법어는 속등록 제18권, 보등록 제5권 드에 수록되어 전하고 있지만 그의 생애와 생목년도에 대해서는 자세히 전하지 않고 있다.

\paragraph{}
속등록에는 1101년 보등록에는 1202년에 선원청규를  만들었다고 전하는 기록으로 볼때, 북송 시대에 활약한 선승이라 추측 할 수 있을 뿐이다.


\paragraph{}t
특히 속등록의 저자인 불국유백은 법운법수 원통선사의 법사인데 , 종색은 처음 법수에게 나아가 출가한 사람이므로, 종색과 유백은 동문임을 알수 있다.
이렇게 볼때 속등록의 기록이 신뢰도는 높은데 속등록에는 종색의 법문과 기연을 수록하고, 그의 전기로 성손씨 낙주영연인 姓孫氏 洛州永年人 이라고 전하고 있을 뿐이다.

\paragraph{}
종색이 스승 법수선사의 문하를 떠나 법수선사의 법형인 장로응부를 참문하여 그의 법을 계승하고, 숭령녕간 崇寧年間 (1102-1106) 진정부홍제선원에 주석하였으며, 아곳에서 선원청규를  편찬하였다. 좌선의도 선원청규에 포함되어 있다.


\paragraph{}

종색은 진정부 홍제선원과 장루 숭복 선원에 주석하며 법을 펼쳤다.
홍제선원에 대해선 잘 알 수 없지만, 진정부는 북쪽 임제가 활약한 정정의 땅인데, 종색은 이곳에서 그가 말년에 입적한 양쯔강 유역의 장로사를 중심으로 주석하면서 행화를 펼쳤을 것으로 간주된다.
그의 구체적인 활동과 입멸년은 알수 없으나 「자각대사慈覺大師 」의 시호를 하사 받은 사실로 볼 때 당시 위대한 선승으로 살았음을 알 수 있다.

\paragraph{ 정토불교 실천사상가 }
또한 종색은 정토불교의 실천사상가로도 잘 알려진 사람이다. 
약방문류의 편자인 종효는 종색을 연종오조의 지위로 부여하고 있다.
정토사상가로서의 종색은 관념불송 (낙방문류 제5권) 『권수선입선인겸수정토』 용서정토문 제11권의 저작이 있다.


\paragraph{}
좌선의와 선원청규가 참선 수행자들을 위한 입문과 승단의 규칙을 기록한 것인데, 
그 가운데 염불의식의 규정이 보이고 있는 점은 종색의 이와 같은 선정융합사상에 토대를 두고 있음을 반영한 것이다.
사실 이러한 선정융합 내지 선정쌍수의 주장은 종색 한 사람 뿐만 아니라 북송시대의 사상 경향이었으며,
종색도 이러한 시대적인 사조를 받아 들인 것이라고 할 수 있다.




\paragraph{문집}
지원집 권하에 장로 색선사 문집서 ( 속장경 105권 302쪽b )가 전하고 있는 것으로 볼때 그의 문집이 있었음을 알 수 있지만, 전하지 않아 그 내용을 알 수 가 없다.


\paragraph{종색선사 법계}
종색선사의 법계를 도시해 보면 다음과 같다.
운문문언
양림진원
지문광조
성두중현
천의의회
장로응부
장로종색
법운법수
불국유백




% ----------------------------------------------------------------------------- 자선의의 전래
%
% -----------------------------------------------------------------------------
	\section{자선의 전래 }


\paragraph{}
종색선사의 좌선의는 그가 편찬한 선원청규 제8권에 귀경문 좌선의 자경문과 함께 수록되어 전하고 있다.
그러나 고려본 선원청규에는 원래 좌선의 가 수록되지 않고 있는 점으로 볼때 선원청규를 재편할 때 편입했음을 알수 있다.

\paragraph{}
좌선의는 선수행을 지향하는 선수행자를 올바르게 좌선실천을 할 수 있도록 하는 수행방법을 제시함과 동시에 개인을 규제하는 최소한의 율법인 것이다.
좌선명이나 좌선잠, 귀경문은 선의 사상을 설한 가장 내면적인 것이지만, 
개인이나 집단에도 구체적인 규제를 미치지 못하고 있다.
이것은 어디까지나 각자 개인의 자기 향상을 위한 정신적인 것이기 때문이다.


\paragraph{}
좌선의가 뒤에 선원청규의 일부로 편입됨으로 인하여 수행자의 정신과 선수행의 실천적 행위의 규범을 준수하게 하는 구체적인 규제의 제일보가 되었다.
이로부터 종색의 좌선의는 선수행 입문서의 대표적인 작품으로 평가되고 있다.


\paragraph{}



\paragraph{참고 그외의 좌선의 }



% ----------------------------------------------------------------------------- 장로 자각 색 선사 좌선의
%
% -----------------------------------------------------------------------------
	\section{장로 자각 색 선사 좌선의 長蘆 慈覺 賾 禪師 坐禪儀 }
          

\paragraph{}
반야를 배우는 보살은 먼저 마땅히 대 자비심을 일으키고 
커다란 서원을 발하여 삼매를 정미롭게 닦으며 
중생 제도를 서약해야 할 것이니 
한 몸 홀로 해탈을 구함이 되지는 말지어다. 
그리고는 모든 인연을 버리고 만 가지 일을 쉬며 
몸과 마음을 한결 같이하여 움직임과 고요함에 간격이 없이하며, 
먹고 마심을 요량하여 적지도 않고 많지도 않게 하고 
수면을 조절하여 너무 절제하지도 말고 너무 내키는대로 하지도 말라.

\paragraph{}
좌선하고자 할 때는 
한가하고 고요한 곳에서 깔개를 두껍게 깔고 
옷의 띠는 느슨하게 매되 위의를 가지런히 한 후에 결가부좌함에, 
먼저 오른 발을 왼쪽 넓적다리 위에 편안히 올려놓고 
왼쪽 발을 오른쪽 넓적다리 위에 편안히 올려놓는다. 
혹은 반가부좌도 괜찮으니, 단지 왼 발로 오른 발을 눌러 줄 따름이다. 

\paragraph{}
다음에는 오른 손을 왼 발 위에 편안히 올려놓고 
왼쪽 손바닥을 오른쪽 손바닥 위에 올려놓은 뒤에 
양손의 엄지손가락을 마주하여 서로 떠받치게 하고는 
서서히 몸을 들어 전방을 향한다. 
다시 좌우로 흔들고는 이에 몸을 바로하고 단정히 앉되
좌우로 기울거나 앞뒤로 굽히지 말아야 하며 
허리의 등골뼈와 머리와 목덜미의 골절을 서로 떠받치게 하여 
그 형상이 마치 부도浮屠와 같아야 한다. 

\paragraph{}
또 몸을 너무 지나치게 솟구침으로써 
호흡의 기운이 급하여 불안하게 하지 말아야 하며, 
귀는 어깨와 더불어 수직이 되게 하고 
코는 배꼽과 더불어 수직이 되게 해야 하며, 
혀는 윗잇몸을 떠받치고 입술과 이는 서로 붙이며, 
눈은 모름지기 가늘게 떠서 
얼핏 잠드는 것을 면하도록 해야 할 것이니, 
만약 선정을 얻었다면 그 힘이 가장 수승할 것이다.

\paragraph{}
옛날에 선정을 익히던 고승이 있었는데 
앉았을 때는 항상 눈을 뜨고 있었으며,
예전에 법운 원통선사 역시 눈을 감고 좌선하는 사람들을 꾸짖어 
그것을 검은 산의 마귀 소굴로 여겼으니
대개 깊은 뜻이 있는지라 통달한 자는 알 것이다.

\paragraph{}
몸의 모습이 이미 안정되고 호흡의 기운이 이미 조절 된 연후에 
배꼽과 배를 느슨히 풀어놓아 
일체의 선과 악을 모두 생각하여 헤아리지 말라. 
망념이 일어나거든 [망념이 일어났음을] 곧 깨달을지니 
그것을 깨달으면 곧 없어질 것이다. 
오래도록 반연하는 바를 잊으면 자연스레 집중을 이룰 것이니 
이것이 좌선의 요긴한 방술이다.

\paragraph{}
가만히 생각건대 좌선은 곧 안락한 법문인데 
사람들이 많이들 질병을 이루는 것은
대개 마음 쓰기를 잘하지 못한 까닭이다. 
만약 이 뜻을 잘 체득하면 곧 자연히 육신이 가볍고도 
편안해 질 것이고 정신이 상쾌하고도 날래게 될 것이며
정념正念이 분명하여 법의 맛이 정신을 도울 것이므로 
고요히 맑고 즐거울 것이다. 
만약 이미 깨달은 바가 있는 자라면 마치 
용이 물을 얻은 것과 같고 
흡사 호랑이가 산을 의지한 것이라 말할 수 있으며, 
만약 아직 깨달음이 있지 않다 하더라도 또한 바람으로 인하여 
불길을 부추기는 것이라 힘씀이 그리 많지 않으리니
다만 긍정적인 마음으로 힘쓰면 반드시 속임을 당하지는 않을 것이다.

\paragraph{}
그러나 道가 높아지면 魔가 왕성하여 
순조로움을 거스르는 경계가 만 가지로 나타날 것이니 
단지 바른 생각이 앞에 드러난다면 
일체의 것이 만류하거나 장애하지 못하리다. 
예컨대《능엄경》과《천태지관》및《규봉수증의》등에서 
마군의 일을 갖추어 밝혀 놓아서 조심하지 못하는 자에게 
예비토록 하고 있으니 불가불 알아야 한다.

\paragraph{}
만일 선정에서 나오고자 한다면 서서히 몸을 움직임에 
편안하고도 자세히 하여 일어나야지 갑작스레 해서는 안되며, 
선정에서 나온 후에는 일체의 시간 중에 항상 방편에 의지하여 
선정의 힘을 보호하여 가지되 
마치 갓난애를 보호하듯 해야 곧 선정의 힘을 쉽게 이룰 것이다.

\paragraph{}
무릇 ‘선정’이라는 이 한 부문이 가장 급선무가 되니, 
만약 편안히 선정에 들어 고요한 생각을 지니지 못하면 
[죽음의] 경계에 이르러 모두 망연해질 뿐이다. 
그러한 까닭으로 구슬을 찾으려면 물결이 고요해야 하니 
물이 움직이면 취하기가 응당 어려울 것이요, 
선정의 물이 고요하고도 맑으면 마음의 구슬은 저절로 드러날 것이다. 
그러므로《원각경》에 이르기를 
「장애 없는 청정한 지혜는 모두 선정에 의지해 생겨난다」 하였고,
《법화경》에 이르기를 
「한가한 곳에 있으면서 그 마음을 닦아 거두어들이되 
편안히 머물어 움직이지 않음이 마치 수미산 같을지다」 하였다.

\paragraph{}
이로서 알건대 범부를 초월하고 성현을 뛰어 넘으려면 
필시 고요함의 반연을 빌릴 것이요, 
좌탈坐脫하고 입망立亡하려면 모름지기 선정의 힘에 의지해야 한다. 
일생 동안에 끝장을 보고자 하더라도 오히려 차질이 날까 두렵거늘 
하물며 이에 미적미적하면 무엇을 가지고 업에 대적하겠는가. 
그러므로 옛 사람이 이르기를 「만약 선정의 힘이 없으면 
죽음의 문에 달갑게 엎드려 눈을 가리고 텅 빈 채 돌아갈 때 
의연히 물결 따라 흘러갈 지어다」 하였다.

\paragraph{}
바라건대 모든 선우禪友들이 이 글을 하루에 세 번 반복하여 읽어서 
스스로를 이롭게 하고 나아가 
다른 이를 이롭게 함으로써 함께 바른 깨달음을 이룰지어다.


\paragraph{}
- 長蘆 慈覺 宗賾禪師'좌선의' / 치문경훈緇門警訓 '선문禪文' 에서 



% ----------------------------------------------------------------------------- 장로자각색선사좌선의
%
% -----------------------------------------------------------------------------
	\section{장로자각색선사좌선의 : 원문}



長蘆慈覺賾禪師坐禪儀

學般若菩薩은 先當 [起大悲心하야 發弘誓願하며 精修三昧하야 誓度衆生이요] 不爲一身하야 獨求解脫이니라

爾乃放捨諸緣하야 休息萬事하고 身心一如하야 動靜無間하며 量其飮食하야 不多不小하고 調其睡眠하야 不節不恣니라

欲坐禪時어든 於閒靜處에 厚敷坐物하고 寬繫衣帶하야 令威儀로 齊整然後에

結跏趺坐호대 先以右足으로 安左髀上하고 左足으로 安右髀上하며 或半跏趺라도 亦可니 但以左足으로 壓右足而已니라 次以右手로 安左足上하고 左掌으로 安右掌上하며 以兩手大拇指面으로 相拄하고 徐徐擧身前向하며 復左右搖振하고 乃正身瑞坐호대 不得[左傾右側하고 前躬後仰하며] 令腰脊頭項骨節로 相拄호대 狀如浮屠하며 又不得[聳身太過하야 令人氣急不安하고

要令耳與肩對하며 鼻與臍對하고 舌拄上齶하야 唇齒相着하며 目須微開하야 免致昏睡니 若得禪定이면 其力이 最勝이니라

古有習定高僧이 坐常開目하고 向에 法雲圓通禪師가 亦訶人閉目坐禪호대 以爲黑山鬼窟이라하니 盖有深旨라 達者는 知焉이니라

身相을 旣定하고 氣息을 旣調然後에 寬放臍腹하야 一切善惡을 都莫思量하며 念起卽覺이니 覺之卽失이라 久久忘緣하면 自成一片하리니此는 坐禪之要術也니라

竊爲坐禪은 乃安樂法門이어늘 而人多致疾者는 盖不善用心故也니라

若善得此意則自然四大輕安하고 精神이 爽利하며 正念이 分明하야 法味資神일새 寂然淸樂하리라

若已有發明者인댄 可謂如龍得水하고 似虎靠山이며 若未有發明者라도 亦乃因風吹火라 用力이 不多하리니 但辦肯心하라 必不相賺이니라

然而道高魔盛하야 逆順萬端이니 但能正念現前하면 一切不能留礙라

如楞嚴經과 天台止觀과 圭峰修證儀에 具明魔事하야 預備不虞者하니 可不知也니라

若欲出定커든 徐徐動身하야 安詳而起하고 不得卒暴하며 出定之後에는 一切時中에 常依方便하야 護持定力호대 如護嬰兒하면 卽定力을 易成矣리라

夫禪定一門이 最爲急務니 若不安禪靜慮하면 到遮裡하야 總須茫然이라

所以로 探珠에 宜靜浪이니 動水하면 取應難이니라 定水澄淸하면 心珠自現이니 故로 圓覺經에 云無礙淸淨慧가 皆依禪定生이라하며 法華經에 云在於閒處하야 修攝其心호대 安住不動을 如須彌山이라하시니

是知超凡越聖인댄 必假靜緣이요 坐脫立亡인댄 須憑定力이니라

一生取辦하야도 尙恐蹉跎온 況乃遷延이면 將何敵業이리요 故로 古人이 云若無定力이면 甘伏死門하야 掩目空歸에 宛然 流浪이라하니

幸諸禪友는 三復斯文하고 自利利他하야 同成正覺이어다

 
% ----------------------------------------------------------------------------- 장로자각색선사좌선의
%
% -----------------------------------------------------------------------------
	\section{장로자각색선사좌선의 : 해설}

 

장로자각색선사 좌선의  

\paragraph{}
반야를 배우는 보살은, 먼저 대비심을 일의켜 큰 서원을 세워 삼매를 세밀하게 닦아 중생을 제도하기를 맹세해야 할것이요, 한몸을 위해 홀로 해탈을 구하지 말지어다.
s
그리하여 온갖인연을 놓아버려 만가지 일을 그치어 쉬고, 몸과마음이 한결같아 움직이고 고요함에 간격이 없어야하며, 마시고 먹는 것을 헤아려서 많지도 않고 적지도 않게 하고, 수면을 조절하여 절제하지도 방자하지도 말지니라. (너무 안자려고도 하지말고, 퍼질러 늘어져 자지도 마라)

좌선하고자 할 때는 한가하고 고요한 곳에서 깔개를 두껍게 펴고, 옷의 허리띠는 느슨하게 매어 위의가 가지런하고 단정하게 한 연후에,

결가부좌하되 먼저 오른발을 왼쪽 넓적다리위에 편하게놓고, 왼발을 오른쪽 넓적다리위에 편안하게 놓으며, 혹은 반가부좌라도 또한 가능하니, 다만 왼발로 오른발을 눌러주기만하면 되느니라. 다음에 오른손으로 왼발위에 편안히놓고, 왼손바닥을 오른손바닥위에 편안히 놓으며, 양손의 엄지손가락면이 마주하여 서로 떠받치게 하고, 천천히 몸을 들어 앞쪽을 향하며, 다시 좌우로 흔들고 그리고나서 몸을 바로하고 단정히 앉되, 왼쪽으로 기울거나 오른쪽으로 기울어서는 않되고, 몸을 앞으로 뒤로 기울어서는 않되며, 허리 등뼈 머리 목덜미의 뼈마디가 서로 떠받치게하되, 모습이 부도와 같이하며, 또한 몸이 너무 지나치게 솟구쳐서 다른사람들의(人) 기운이 급하고 불안하게 하지말고,

반드시 귀와 어깨가 곧게하고, 코와 배꼽이 곧게해야하며, 혀는 윗임몸으로 떠받쳐서 입술과 이가 서로 붙으며, 눈은 모름지기 작게 열어서 어두워서 잠들지 않도록 해야하니, 만약 선정을 얻으면 그힘이 가장 뛰어날 것이니라.

옛날에 선정을 익히던 뛰어난스님이 있었는데, 않으면 항상 눈을떳고, 예전에(向) 법운원통선사가 또한 사람들이 눈을감고 좌선하는 것을 꾸짖으면서 암흑산의 귀신굴이라하니, 대개 깊은 맛이 있음에 다다른사람은 알것이니라.

몸의 모습이 이미 안정하고 기운과호흡을 조절하고난 연후에배꼽과 배를 느슨하게 풀어놓아, 일체의 선악을 조금도 생각하고헤아리지말며,망념이 일어나면 알아챌지니, 알아채면 없어질 것이다.변함없이 오랫동안 연을 잊으면, 저절로 한조각 이루게 되니,이것은 좌선의 요긴한 기술이니라.

사실 좌선을 하는 것이 바로 안락법문인데도, 사람들이 질병에 이르는 자가 많은 것은 대개 마음쓰기를 잘하지 못하기 때문이니라.

만약 이 뜻을 잘 알면 자연히 사대가 가볍고 평안하고,정신이 맑게통하며, 정념이 분명해져서 법의 맛이 정신의 밑거름일새,고요하면서 맑고 즐거우리라.

만약 이미 밝게일어난 바가있는자라면 마치 용이 물을얻은것과같고, 호랑이가 산을 의지한것과 같다고 말할 수 있으며, 만약 아직 밝게일어난 바가 없는자라도 또한 이에 바람으로 인하여 불을 부추김이라. 힘을 쓰는 것을 많이 않하리니, 다만 긍정적인 마음으로 힘쓰라. 반드시 서로 속이는 것이 없을것이니라.

그럼에도 도가높아지면 마도 치성하야 도리를 거스리는경계들이 만가지가되니(많아진다)다만 정념이 현전하게만 할수있다면, 일체가(그 어떤것도) 지체하거나 장애하지 못할 것이다.

마치 능엄경과 천태지관과 규봉수증의에서 모두가 마구니일을 밝히어, 헤아리지 못하는 자들이 미리 갖출수있도록하니, 반드시알수있을 것이다.(이중부정 강조)

만일 선정에서 나오려거든 천천히 몸을 움지여서, 차분하고 세밀히하여 일어나야하고, 갑작스럽게하면 안되며, 선정에서 나온후에는 24시간 항상 방편에 의지하여 선정의 힘을 보호하여지키되, 갓난아이 보호하듯하면, 선정의힘을 쉽게 이루리라.

대저 선정이라는 한가지 문(방법)이 가장 시급한 일이니,만약 차분하게 선정에들어서 맑은생각을 못하면 다스림이 막혀서 단속하는 것이 아득해지게 될뿐이라.

그런까닭에 구슬을 찾으려면 마땅히 물결을 고요해야하니, 물이 움직이면, 취하기가 응당 어려우니라. 선정의 물이 맑고 깨끗하면, 마음구슬이 저절로 나타나니, 그러므로 <원각경>에 이르되, 장애없는 청정한 지혜가 모두 선정에 의지하여 일어남이라 하며,<법화경>에 이르되, 한가한곳에 있어서 그마음을 닦아거두어들이되, 차분히 머물러 움직이지 않기를 수미산같이 할지니라 하시니,

그러므로 알라. 범부를 넘어 성현을 넘어서려면, 반드시 고요한 인연을 빌려야함이요, 좌탈입망하려면 모름지기 선정의 힙에 의지해야하느니라.

한생에 갖추려하여도 오히려 잘못될까 두렵거늘, 하물며 그렇게 느그적거리면 앞으로 어떻게 업에 대적하겠는가. 그러므로 옛사람이 이르기를 만약 선정의힘이 없으면, 죽음의 문에서 즐겁게(쉽게) 굴복하여, 눈가리고 빈손으로 돌아감에 굽이굽이 흘러가리라하니,

원컨대, 모든 선의 도반들은 이글을 세 번 반복하여, 스스로이롭게하고 남도이롭게하야 함께 바른깨달음을 이룰지어다.


% ----------------------------------------------------------------------------- 장로자각색선사좌선의
%
% -----------------------------------------------------------------------------
	\section{坐禪儀(좌선하는 법) - 종색 선사 }

 제주불교신문 승인 2019.11.13 11:35 

\paragraph{}
도내 사찰들이 동안거입제식을 하고 본격적인 안거에 들어갔다. 이에 종색선사의 좌선법을 소개한다.  종색 선사는 중국 송나라 때 스님. 운문종의 법운법수(法雲法秀) 스님에게 출가해서 장로응부(長蘆應夫)의 법을 이어받았다. 1102년부터 1105년경에는 하북성 홍제선원(洪濟禪院)과 장로사(長蘆寺)에 머물기도 했다. 특히 홍제선원의 주지로 있을 당시 선종 사원의 독자적인 계율이라고도 할 수 있는 <선원청규(禪院淸規>를 저술했다. 종색선사 좌선법 다음으로는 간화선 좌선법도 소개한다. /편집자 주

 

\paragraph{좌선하는 자세}
지혜를 배우는 사람들은 먼저 큰 자비심을 일으키고 넓은 서원을 세워서 정미롭게 삼매(三昧)를 닦아야 한다. 중생을 제도하고자 서원(誓願)하고 내 한 몸만을 위해 해탈(解脫)을 구해서는 안 된다.
모든 인연을 놓아 버리고 만사를 쉬어, 몸과 마음이 하나 같고 움직이고 고요함에 틈이 없어야 한다. 음식의 양을 헤아려 너무 배부르거나 배고프지 않게 하고, 잠을 조절하여 모자라거나 지나치게 하지 말라.
좌선을 할 때는 고요한 곳에서 두터운 방석을 깔고 하라. 허리띠는 느슨하게 매고, 몸가짐을 단정히 한 후에 결가부좌(結跏趺坐)를 한다.
오른쪽 발을 왼쪽 넓적다리 위에 놓고, 왼쪽 발을 오른쪽 넓적다리 위에 놓는다. 반(半) 가부좌를 하는 것도 무방하지만 이때 왼쪽 발을 오른쪽 넓적다리 위에 올려놓는다.
다음으로 오른쪽 손을 왼쪽 발 위에 놓고, 왼쪽 손등을 바른쪽 손바닥 위에 놓는다. 두 엄지손가락 끝을 서로 맞대고 서서히 허리를 편 다음 전후 좌우로 몇 번 움직여서 몸을 바르게 하고 단정히 앉는다.
왼쪽으로 기울거나 오른쪽으로 기울거나 앞으로 구부리거나 뒤로 넘어가게도 하지 말고, 허리와 척추, 머리와 목을 똑바로 세워 그 모양이 부도(浮屠)와 같게 한다. 이 때 몸을 너무 긴장시켜 호흡을 부자연스럽게 하는 일이 없어야 한다. 귀와 어깨는 가지런히 하고, 코와 배꼽을 일직선상에 두며, 혀는 입천장에 대고 입을 다문다. 눈은 반만 떠서 졸음에 빠지지 않도록 한다. 이와 같이 해서 선정을 얻으면 그 힘이 크게 넘칠 것이다.

\paragraph{눈을 감지 말라}
옛날 선정(禪定)을 닦던 스님들은 앉아서 항상 눈을 떴으며, 법운원통(法雲圓通) 선사도 눈을 감고 좌선하는 사람들을 꾸짖기를 ‘깜깜한 산의 귀신굴이 된다.’ 고 하였다. 여기에 깊은 뜻이 있으니 통달한 사람은 알 것이다.
자세가 안정되고 호흡이 조절된 다음에는 아랫배에 지긋이 힘을 주고, 일체의 선악을 생각하지 말라. 잡념이 일어나면 거기에서 곧 깨어날 것이니 깨어나면 곧 사라질 것이다. 오래도록 인연을 잊으면 저절로 조금 이루어질 것이니, 이것이 좌선의 요긴한 비법이다.

\paragraph{안락의 법문}
곰곰이 생각하면 좌선은 안락(安樂)의 법문이지만, 사람들이 흔히 병을 얻는 것은 모두 마음을 잘못 쓰기 때문이다. 이 뜻을 잘 터득하면, 자연히 온몸이 편안하고 정신이 상쾌해질 것이다. 바른 생각이 분명하고 법의 맛이 정신을 도와 고요하고 맑은 기쁨을 누릴 것이다. 한번 밝게 된 사람이라면 용이 물을 얻은 것 같고, 호랑이가 산을 의지한 것과 같을 것이다. 아직 밝게 되지 못한 사람은 바람에 의해서만 불을 일으키려는 것과 같아서 그 힘이 모자랄 것이다. 즐거운 마음으로 판단하고 절대로 서로 속이지 말라.
도가 높아지면 마(魔)가 성하는 법이어서 역경과 순탄함이 만 가지나 된다. 그러나 바른 생각이 나타나면 그 어떤 것에도 거리끼지 않을 것이다.
‘능엄경’과 ‘천태지관’과 규봉의 ‘수증의(修證儀)’에 악마의 일을 두루 밝혀, 헤아리지 못하는 사람들을 위해 예비해 두었으니 반드시 알아두어라.

\paragraph{좌선이 끝났을 때}
좌선이 끝나 일어설 때는 천천히 몸을 움직인 다음에 편안히 일어나고 갑자기 일어서지 말라. 좌선에서 일어난 뒤에는 어느 때나 항상 좌선의 방법에 의하여 선정(禪定)의 힘을 보호하고 유지하기를 어린애를 돌보듯 하라. 그러면 선정의 힘을 쉽게 이룰 수 있을 것이다.
이 선정의 한 문이 가장 급한 일이다. 만약 선정을 잘 이루지 못하면 여기에서는 모든 것이 망망할 것이다. 그렇기 때문에 구슬을 찾으려면 물결이 가라앉아야 한다. 물결이 일렁이면 찾기 어렵다. 물결이 가라앉아 맑고 깨끗해지면 마음의 구슬이 저절로 나타난다.
‘원각경’에 이르기를 ‘거리낌없는 청정한 지혜가 다 선정에서 나온다.’ 고 하였고, 법화경에서는 ‘고요한 곳에서 마음을 닦고, 편안히 머물러 움직이지 않기를 수미산처럼 하라.’ 고 하였다.
범부와 성인을 뛰어 넘으려면 반드시 반연을 고요히 하고, 앉아서 가고 서서 가려면 선정의 힘에 의지해야 한다. 한평생 힘을 기울여도 오히려 잘못될까 두려운데, 하물며 게을러 가지고야 어떻게 생사의 업을 막아내겠는가.
그러므로 옛 사람이 이르기를 ‘만약 선정의 힘이 없으면 죽음의 문에 굴복당하고, 눈앞이 캄캄하여 갈팡질팡 헤매게 될 것이다.’ 라고 하였다. 바라건대, 모든 참선하는 벗들은 이 글을 거듭거듭 읽고, 나도 이롭고 남도 이롭게 하여 다 같이 바른 깨달음을 이룰지어다. 

\paragraph{<간화선> 좌선법 }
좌선을 하려면 조용하고 정갈한 곳이 좋다. 그러나 보다 더 중요한 점은 지나치게 장소나 환경에 집착하지 않는 마음가짐이다. 달마 선사는 “밖으로 모든 인연을 끊고 안으로 헐떡거림이 없어 마음이 장벽과 같이 되어야 가히 도에 들어 간다”고 하셨다. 육조 혜능 선사는 『육조단경』에서 “밖으로 모든 경계에 마음이 움직이지 않는 것을 좌坐라 하고 안으로 본래 성품을 보아 어지럽지 않는 것이 선禪이다”라고 하셨다. 참으로 조사스님들의 고구정녕하신 가르침이다. 
좌선법은 먼저 큰 서원을 세워야 한다. 
-바른 법에 대한 신심이 견고하여 영원히 물러나지 않겠다. 
-나고 죽는 생사윤회에서 벗어나 결정코 본래 면목을 깨달으리라. 
-반드시 부처님의 혜명을 잇고 모든 중생을 다 제도하리라. 
이러한 원력을 양식 삼아 좌선할 때만이라도 모든 반연을 놓아 버리고 화두를 면밀히 참구해야 한다.
1. 좌선하는 방법에는 결가부좌와 반가부좌가 있다. 결가부좌는 오른쪽 다리를 왼쪽 허벅지 위에 올려놓고, 왼쪽 다리를 오른쪽 허벅지 위에 올려놓는 자세다. 이때 두 다리를 허벅지 깊숙이 올려놓아야 자세도 안정되며 오래할 수 있다. 반가부좌는 좌복 위에 앉아 왼쪽 다리를 오른쪽 다리 위에 올려놓거나(길상좌) 오른쪽 다리를 왼쪽 다리 위에 올려놓는다(항마좌).  
2. 허리를 자연스럽게 반듯이 세우고 양쪽 어깨에 힘이 들어가지 않도록 한다. 양쪽 귀와 어깨를 나란히 하고 코와 배꼽이 수직이 되도록 한다.   
3. 손은 길상좌일 경우 오른 손바닥을 왼발 위 단전 앞에 자연스럽게 놓고 그 위에 왼 손바닥을 포개어 얹는다. 양쪽 엄지를 가볍게 서로 닿게 붙인다(법계정인). 항마좌인 경우 그 반대로 하면 된다.   
4. 입과 이는 긴장을 풀고 살짝 다물며 혀를 말아 혓바닥 아래쪽이 입천장에 닿도록 한다. 눈은 반쯤 뜨되 부릅뜨지도 말고 감지도 말고 너무나 자연스럽게 마치 머리가 없는 것처럼 생각하고 1~2 미터 앞바닥에 시선을 내려놓는다.  
5. 음식을 너무 많이 먹지 말고 약간 부족한 듯하게 하라. 허리끈은 여유 있게 하고 가능한 말을 많이 하지 말며 모든 긴장을 풀어버리도록 하라.  
6. 호흡은 지극히 자연스럽게 하라. 약간 깊이 들이 마시고 천천히 내쉰다는 생각으로 하되 너무 신경 쓰지 말고 화두만 참구하라.  
7. 몸과 마음을 통째로 화두에 바쳐 버렸다는 마음가짐으로 온통 화두와 하나 되어야 한다. 좌선이 잘 된다는 생각도 잘 안 된다는 생각도 모두 망상이니 오직 화두 참구만 애써 노력하라. 간절하고 진솔하게 하되 속효심도 해태심도 내지 말라.  
8. 경책(警策) - 좌선 중에 졸거나 정신을 집중하지 않아 자세가 흐트러지면 죽비로 경책을 한다. 경책은 바른 수행을 돕는 문수보살의 가르침이다. 경책을 할 때는 소임자가 경책 받을 사람의 오른쪽 어깨 위에 죽비를 가볍게 올려놓고 지그시 누르면서 경책할 것을 알린다. 그러면 경책 받을 이는 졸음에서 깨어 합장하고 머리를 왼쪽으로 가볍게 기울여 어깨로 경책 받도록 한다. 경책 받은 다음에도 합장하여 감사의 인사를 하고 다시 바른 자세로 되돌아간다.  
9. 좌선 시간은 50분 앉았다가 10분 포행하는 게 기본이지만 너무 시간에 구속되지 않아야 한다. 포행은 방선放禪 시간에 선방 안팎을 천천히 걸으면서 다리를 풀어 주는 것을 말한다. 포행 시에도 화두를 놓아서는 안된다.  
 

저작권자 © 제주불교신문 무단전재 및 재배포 금지


%	================================================================== Part			서장
	\addtocontents{toc}{\protect\newpage}
	\part{서장}
	\noptcrule
	\parttoc				


%	================================================================== chapter		강의 구성
%	\addtocontents{toc}{\protect\newpage}
	\chapter{강의 구성}
	\noptcrule

	\newpage	
	\minitoc


% ----------------------------------------------------------------------------- 강의시간
%
% -----------------------------------------------------------------------------
	\section{강의 시간}

% ------------------------------------------------------------------------------ table 																											
\begin{table} [h]																											
\caption{강의시간}
\label{tab:title}																											
\tabulinesep= 			1	em																							
% \tabulinesep= 0.0 em																											
\begin{tabu} to 1.0\linewidth {																											
	X [	r	,	0.10	]		%	1	번호																		
	X [	r	,	2.00	]		%	2	날짜																		
	X [	r	,	3.00	]		%	3	내용
	X [	r	,	1.00	]		%	4	출석
	X [	r	,	1.00	]		%	5	비고
}																											
\hline	\hline														 												
%	1		2		3		4			5			6		7												
No.		&	일자						&	내용					&	출석			&	비고 	\\  \hline \hline											
	1	&	2020년06월15일 월		&	강의 내용 구성 소개 	&	출	 		&			\\  \hline											
	2	&	2020년06월22일 월		&	강의 내용 구성 소개 	&		 		&			\\  \hline											
	3	&	2020년06월29일 월		&	참선 방법			&		 		&			\\  \hline											
	4	&	2020년07월06일 월		&	강의 내용 구성 소개 	&		 		&			\\  \hline											
	5	&	2020년07월13일 월		&	강의 내용 구성 소개 	&		 		&			\\  \hline											
	6	&	2020년07월20일 월		&	강의 내용 구성 소개 	&		 		&			\\  \hline											
	7	&	2020년07월27일 월		&	강의 내용 구성 소개 	&		 		&			\\  \hline											
\end{tabu}																											
\end{table}																											
% ===== ===== ===== ===== ===== ===== ===== ===== .																											
\clearpage																											
% ===== ===== ===== ===== ===== ===== ===== ===== .																											



% ----------------------------------------------------------------------------- 교육 책자
%
% -----------------------------------------------------------------------------
	\section{교육 책자}



% ----------------------------------------------------------------------------- 도반
%
% -----------------------------------------------------------------------------
	\section{도반}


% ------------------------------------------------------------------------------ table 																											
\begin{table} [h]																											
\caption{도반}
\label{tab:title}																											
\tabulinesep= 			1	em																							
% \tabulinesep= 0.0 em																											
\begin{tabu} to 1.0\linewidth {																											
	X [	r	,	0.10	]		%	1	번호																		
	X [	r	,	2.00	]		%	2	날짜																		
	X [	r	,	1.00	]		%	3	내용
	X [	r	,	2.00	]		%	4	출석
	X [	r	,	1.00	]		%	5	비고
}																											
\hline	\hline														 												
%	1		2		3		4			5			6		7												
No.		&	성명				&성별	&	전화번호			&	비고 	\\  \hline \hline											
	1	&	보조 김대희 		&남		&	010.3839.5609	&			\\  \hline											
	2	&	보조 김대희 		&남		&	010.3839.5609	&			\\  \hline											
	3	&	보조 김대희 		&남		&	010.3839.5609	&			\\  \hline											
	4	&	보조 김대희 		&남		&	010.3839.5609	&			\\  \hline											
	5	&	보조 김대희 		&남		&	010.3839.5609	&			\\  \hline											
	6	&	보조 김대희 		&남		&	010.3839.5609	&			\\  \hline											
\end{tabu}																											
\end{table}																											
% ===== ===== ===== ===== ===== ===== ===== ===== .																											
\clearpage																											
% ===== ===== ===== ===== ===== ===== ===== ===== .																											

%	================================================================== chapter		대혜 선사
%	\addtocontents{toc}{\protect\newpage}
	\chapter{대혜 선사}
	\noptcrule

	\newpage	
	\minitoc

% ----------------------------------------------------------------------------- 1
%
% -----------------------------------------------------------------------------
	\section{대혜 선사 행장 }

1089 ~ 1163 선사는 선주 영국현 사람이며 성은 해씨다.

어머니의 꿈에 신비한 사람이 스님 한 분을 모시고 왔는데 얼굴은 검고 코는 우뚝하였다.




%	================================================================== chapter		책내용
%	\addtocontents{toc}{\protect\newpage}
	\chapter{책내용 }
	\noptcrule

	\newpage	
	\minitoc


% ----------------------------------------------------------------------------- 1
%
% -----------------------------------------------------------------------------
	\section{1. 증시랑 천유가 질문한 편지}
% ----------------------------------------------------------------------------- 2
%
% -----------------------------------------------------------------------------
	\section{2. 증시랑 천유에게 보낸 답장 1 }

% ----------------------------------------------------------------------------- 3
%
% -----------------------------------------------------------------------------
	\section{3. 증시랑 천유에게 보낸 답장 2 }

% ----------------------------------------------------------------------------- 4
%
% -----------------------------------------------------------------------------
	\section{4. 증시랑 천유에게 보낸 답장 3 }

% ----------------------------------------------------------------------------- 5
%
% -----------------------------------------------------------------------------
	\section{5. 증시랑 천유에게 보낸 답장 4 }

% ----------------------------------------------------------------------------- 6
%
% -----------------------------------------------------------------------------
	\section{6. 증시랑 천유에게 보낸 답장 5 }

% ----------------------------------------------------------------------------- 7
%
% -----------------------------------------------------------------------------
	\section{7. 증시랑 천유에게 보낸 답장 6 }

% ----------------------------------------------------------------------------- 8
%
% -----------------------------------------------------------------------------
	\section{8. 이참정 한로가 질문한 편지 1}

% ----------------------------------------------------------------------------- 9
%
% -----------------------------------------------------------------------------
	\section{9. 이참정 한로에게 보낸 답장 1}

% ----------------------------------------------------------------------------- 10
%
% -----------------------------------------------------------------------------
	\section{10. 이참정 한로가 다시 묻는 편지 2}

% ----------------------------------------------------------------------------- 11
%
% -----------------------------------------------------------------------------
	\section{11. 이참정 한로에게 보낸 답장 2 }

% ----------------------------------------------------------------------------- 12
%
% -----------------------------------------------------------------------------
	\section{12. 강급사 소명에게 보낸 답장 }

% ----------------------------------------------------------------------------- 13
%
% -----------------------------------------------------------------------------
	\section{13. 부추밀 계신에게 보낸 답장 1}

% ----------------------------------------------------------------------------- 14
%
% -----------------------------------------------------------------------------
	\section{14. 부추밀 계신에게 보낸 답장 2}

% ----------------------------------------------------------------------------- 15
%
% -----------------------------------------------------------------------------
	\section{15. 부추밀 계신에게 보낸 답장 3}


% ----------------------------------------------------------------------------- 3
%
% -----------------------------------------------------------------------------
	\section{16. 이참정 한로에게 보낸 별도의 편지 }

% ----------------------------------------------------------------------------- 3
%
% -----------------------------------------------------------------------------
	\section{17. 진소경 계임에게 보낸 답장1 }

% ----------------------------------------------------------------------------- 3
%
% -----------------------------------------------------------------------------
	\section{18. 진소경 계임에게 보낸 답장 2 }

% ----------------------------------------------------------------------------- 3
%
% -----------------------------------------------------------------------------
	\section{19. 조대제 도부에게 보낸 답장 }

% ----------------------------------------------------------------------------- 3
%
% -----------------------------------------------------------------------------
	\section{20. 허사리 수원에게 보낸 답장 }

% ----------------------------------------------------------------------------- 3
%
% -----------------------------------------------------------------------------
	\section{21. 허사리 수원에게 보낸 답장 2 }

% ----------------------------------------------------------------------------- 3
%
% -----------------------------------------------------------------------------
	\section{22. 유보학 언수에게 보낸 답장 }

% ----------------------------------------------------------------------------- 3
%
% -----------------------------------------------------------------------------
	\section{23. 유통판 언충에게 보낸 답장 1}

% ----------------------------------------------------------------------------- 3
%
% -----------------------------------------------------------------------------
	\section{24. 유통판 언충에게 보낸 답장 2}

% ----------------------------------------------------------------------------- 3
%
% -----------------------------------------------------------------------------
	\section{25. 진국태 부인에세 보낸 답장 }

% ----------------------------------------------------------------------------- 3
%
% -----------------------------------------------------------------------------
	\section{26. 장승상 덕원에게 보낸 답장 }

% ----------------------------------------------------------------------------- 3
%
% -----------------------------------------------------------------------------
	\section{27. 장제형 양숙에게 보낸 답장 }

% ----------------------------------------------------------------------------- 3
%
% -----------------------------------------------------------------------------
	\section{28. 왕내한 언장에게 보낸 답장 1 }

% ----------------------------------------------------------------------------- 3
%
% -----------------------------------------------------------------------------
	\section{29. 왕내한 언장에게 보낸 답장 2 }

% ----------------------------------------------------------------------------- 3
%
% -----------------------------------------------------------------------------
	\section{30. 왕내한 언장에게 보낸 답장 3 }

% ----------------------------------------------------------------------------- 3
%
% -----------------------------------------------------------------------------
	\section{31. 하운사에게 보낸 답장 }

% ----------------------------------------------------------------------------- 3
%
% -----------------------------------------------------------------------------
	\section{32. 여사인 거인에게 보낸 답장 1}

% ----------------------------------------------------------------------------- 3
%
% -----------------------------------------------------------------------------
	\section{33. 여낭중 융례에게 보낸 답장 }

% ----------------------------------------------------------------------------- 3
%
% -----------------------------------------------------------------------------
 	\section{34. 여사인 거인에게 보낸 답장 2 }

% ----------------------------------------------------------------------------- 3
%
% -----------------------------------------------------------------------------
 	\section{35. 여사인 거인에게 보낸 답장 3 }

% ----------------------------------------------------------------------------- 3
%
% -----------------------------------------------------------------------------
 	\section{36. 왕장원 성석에게 보낸 답장 1 }

% ----------------------------------------------------------------------------- 3
%
% -----------------------------------------------------------------------------
 	\section{37. 왕장원 성석에게 보낸 답장 2 }

% ----------------------------------------------------------------------------- 3
%
% -----------------------------------------------------------------------------
 	\section{38. 종직각에게 보낸 답장 }

% ----------------------------------------------------------------------------- 39
%
% -----------------------------------------------------------------------------
 	\section{39. 이참정 태발에게 보낸 답장 }

% ----------------------------------------------------------------------------- 40
%
% -----------------------------------------------------------------------------
 	\section{40. 증종승 천은에게 보낸 답장 }

% ----------------------------------------------------------------------------- 40
%
% -----------------------------------------------------------------------------
 	\section{41. 왕교수 대수에ㅅ게 보낸 답장 }

% ----------------------------------------------------------------------------- 40
%
% -----------------------------------------------------------------------------
 	\section{42. 유시랑 계고에게 보낸 답장 1 }

% ----------------------------------------------------------------------------- 40
%
% -----------------------------------------------------------------------------
 	\section{43. 유시랑 계고에게 보낸 답장 2 }

% ----------------------------------------------------------------------------- 40
%
% -----------------------------------------------------------------------------
 	\section{44. 이낭중 사표에게 보낸 답장 }

% ----------------------------------------------------------------------------- 40
%
% -----------------------------------------------------------------------------
 	\section{45. 이보문 무가에게 보낸 답장 }

% ----------------------------------------------------------------------------- 40
%
% -----------------------------------------------------------------------------
 	\section{46. 향시랑 백공에게 보낸 답장 }

% ----------------------------------------------------------------------------- 40
%
% -----------------------------------------------------------------------------
 	\section{47. 진교수 부경에게 보낸 답장 }

% ----------------------------------------------------------------------------- 40
%
% -----------------------------------------------------------------------------
 	\section{48. 임판원 소첨에게 보낸 답장 }

% ----------------------------------------------------------------------------- 40
%
% -----------------------------------------------------------------------------
 	\section{49. 황지현 자여에게 보낸 답장 }

% ----------------------------------------------------------------------------- 50
%
% -----------------------------------------------------------------------------
 	\section{50. 엄교수 자경에게 보낸 답장 } 

% ----------------------------------------------------------------------------- 50
%
% -----------------------------------------------------------------------------
 	\section{51. 장시랑 자소에게 보낸 답장 }

% ----------------------------------------------------------------------------- 50
%
% -----------------------------------------------------------------------------
 	\section{52. 서현모 치산에게 보낸 답장 }

% ----------------------------------------------------------------------------- 50
%
% -----------------------------------------------------------------------------
 	\section{53. 양교수 언후에게 보낸 답장 }

% ----------------------------------------------------------------------------- 50
%
% -----------------------------------------------------------------------------
 	\section{54. 누추밀 중훈에게 보낸 답장 1 }

% ----------------------------------------------------------------------------- 50
%
% -----------------------------------------------------------------------------
 	\section{55. 누추밀 중훈에게 보낸 답장 2 }

% ----------------------------------------------------------------------------- 50
%
% -----------------------------------------------------------------------------
 	\section{56. 조태위 공현에게 보낸 답장 }


% ----------------------------------------------------------------------------- 50
%
% -----------------------------------------------------------------------------
 	\section{57. 영시랑 무실에게 보낸 답장 1 }

% ----------------------------------------------------------------------------- 50
%
% -----------------------------------------------------------------------------
 	\section{58. 영시랑 무실에게 보낸 답장 2 }

% ----------------------------------------------------------------------------- 50
%
% -----------------------------------------------------------------------------
 	\section{59. 황문사 절부에게 보낸 답장 }

% ----------------------------------------------------------------------------- 60
%
% -----------------------------------------------------------------------------
 	\section{60. 손지현에게 보낸 답장 }

% ----------------------------------------------------------------------------- 61
%
% -----------------------------------------------------------------------------
 	\section{61. 장사인 장원에게 보낸 답장 }

% ----------------------------------------------------------------------------- 62
%
% -----------------------------------------------------------------------------
 	\section{62. 탕승상 전지에게 보낸 답장 }

% ----------------------------------------------------------------------------- 63
%
% -----------------------------------------------------------------------------
 	\section{63. 변제형 무실에게 보낸 답장 }

% ----------------------------------------------------------------------------- 64
%
% -----------------------------------------------------------------------------
 	\section{64. 성천규 화상에게 보낸 답장 }

% ----------------------------------------------------------------------------- 65
%
% -----------------------------------------------------------------------------
 	\section{65. 고산체 장로에게 보낸 답장 }












%	================================================================== chapter		강의 
	\addtocontents{toc}{\protect\newpage}
	\chapter{강의 }
	\noptcrule

	\newpage	
	\minitoc


% ----------------------------------------------------------------------------- 강의 2020.06.16
%
% -----------------------------------------------------------------------------
	\section{2020년 6월 16일 : 1강}

		\subsection{강의 내용}

		\subsection{반성문}

\paragraph{	준비물 부족}
			\begin{itemize}	[
							topsep=0.0em, 
							parsep=0.0em, 
							itemsep=0em, 
							leftmargin=6.0em, 
							labelwidth=3em, 
							labelsep=3em
							] 
			\item 커피
			\item 필기용 노트
			\item 교재 제본
			\end{itemize}


\paragraph{교재}
			\begin{itemize}	[
							topsep=0.0em, 
							parsep=0.0em, 
							itemsep=0em, 
							leftmargin=6.0em, 
							labelwidth=3em, 
							labelsep=3em
							] 
			\item 찬불가
			\item 노트용 빈페이지
			\item 
			\end{itemize}



% ----------------------------------------------------------------------------- 강의 2020.06.22
%
% -----------------------------------------------------------------------------
	\section{2020년 6월 22일 : 2}

		\subsection{강의 내용}

		\subsection{반성문}


% ----------------------------------------------------------------------------- 강의 2020.06.29
%
% -----------------------------------------------------------------------------
	\section{2020년 6월 29일 : 3}

		\subsection{강의 내용}

		\subsection{반성문}


% ----------------------------------------------------------------------------- 강의 2020.07.06
%
% -----------------------------------------------------------------------------
	\section{2020년 7월 06일 : 4}

		\subsection{강의 내용}

		\subsection{반성문}


% ----------------------------------------------------------------------------- 강의 2020.07.13
%
% -----------------------------------------------------------------------------
	\section{2020년 7월 13일 : 5}

		\subsection{강의 내용}

		\subsection{반성문}


% ----------------------------------------------------------------------------- 강의 2020.07.20
%
% -----------------------------------------------------------------------------
	\section{2020년 7월 20일 : 6}

		\subsection{강의 내용}

		\subsection{반성문}


% ----------------------------------------------------------------------------- 강의 2020.07.27
%
% -----------------------------------------------------------------------------
	\section{2020년 7월 27일 : 7}

		\subsection{강의 내용}

		\subsection{반성문}


% ----------------------------------------------------------------------------- 강의 2020.08.3
%
% -----------------------------------------------------------------------------
	\section{2020년 8월 3일 : 8}

		\subsection{강의 내용}
		여름 방학

		\subsection{반성문}


% ----------------------------------------------------------------------------- 강의 2020.08.10
%
% -----------------------------------------------------------------------------
	\section{2020년 8월 10일 : 9}

		\subsection{강의 내용}

		\subsection{반성문}


% ----------------------------------------------------------------------------- 강의 2020.08.17
%
% -----------------------------------------------------------------------------
	\section{2020년 8월 17일 : 10}

		\subsection{강의 내용}

		\subsection{반성문}

% ----------------------------------------------------------------------------- 강의 2020.08.24
%
% -----------------------------------------------------------------------------
	\section{2020년 8월 24일 : 11}

		\subsection{강의 내용}

		\subsection{반성문}


% ----------------------------------------------------------------------------- 강의 2020.08.31
%
% -----------------------------------------------------------------------------
	\section{2020년 8월 31일 : 12}

		\subsection{강의 내용}

		\subsection{반성문}





% ------------------------------------------------------------------------------
% End document
% ------------------------------------------------------------------------------
\end{document}


	\href{https://www.youtube.com/watch?v=SpqKCQZQBcc}{태양경배자세A}
	\href{https://www.youtube.com/watch?v=CL3czAIUDFY}{태양경배자세A}


https://docs.google.com/spreadsheets/d/1-wRuFU1OReWrtxkhaw9uh5mxouNYRP8YFgykMh2G_8c/edit#gid=0
+

https://seoyeongcokr-my.sharepoint.com/:f:/g/personal/02017_seoyoungeng_com/Ev8nnOI89D1LnYu90SGaVj0BTuckQ46vQe1HiVv-R4qeqQ?e=S3iAHi