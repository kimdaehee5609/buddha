%	------------------------------------------------------------------------------
%
%	작성 : 2020년 
%			7월 
%			22일 
%			첫 작업
%	
%	천수경에서 떼어내서 따로 한페이로 편집	
%    

%	\documentclass[25pt, a1paper]{tikzposter}
%	\documentclass[25pt, a0paper, landscape]{tikzposter}
	\documentclass[25pt, a2paper ]{tikzposter}

%	\documentclass[25pt, a1paper]{tikzposter}
%	\documentclass[25pt, a1paper]{tikzposter}
%	\documentclass[25pt, a1paper]{tikzposter}

% 	12pt  14pt 17pt  20pt  25pt
%
%	a0 a1 a2
%
%	landscape  portrait
%

	%% Tikzposter is highly customizable: please see
	%% https://bitbucket.org/surmann/tikzposter/downloads/styleguide.pdf

	%	========================================================== 	Package
		\usepackage{kotex}						% 한글 사용


%% Available themes: see also
%% https://bitbucket.org/surmann/tikzposter/downloads/themes.pdf
%	\usetheme{Default}
%	\usetheme{Rays}
%	\usetheme{Basic}
	\usetheme{Simple}
%	\usetheme{Envelope}
%	\usetheme{Wave}
%	\usetheme{Board}
%	\usetheme{Autumn}
%	\usetheme{Desert}

%% Further changes to the title etc is possible
%	\usetitlestyle{Default}			%
%	\usetitlestyle{Basic}				%
%	\usetitlestyle{Empty}				%
%	\usetitlestyle{Filled}				%
%	\usetitlestyle{Envelope}			%
%	\usetitlestyle{Wave}				%
%	\usetitlestyle{verticalShading}	%


%	\usebackgroundstyle{Default}
%	\usebackgroundstyle{Rays}
%	\usebackgroundstyle{VerticalGradation}
%	\usebackgroundstyle{BottomVerticalGradation}
%	\usebackgroundstyle{Empty}

%	\useblockstyle{Default}
%	\useblockstyle{Basic}
%	\useblockstyle{Minimal}		% 이것은 간단함
%	\useblockstyle{Envelope}		% 
%	\useblockstyle{Corner}		% 사각형
%	\useblockstyle{Slide}			%	띠모양  
	\useblockstyle{TornOut}		% 손그림모양


	\usenotestyle{Default}
%	\usenotestyle{Corner}
%	\usenotestyle{VerticalShading}
%	\usenotestyle{Sticky}

%	\usepackage{fontspec}
%	\setmainfont{FreeSerif}
%	\setsansfont{FreeSans}

%	------------------------------------------------------------------------------ 제목

	\title{ 마하 반야 바라밀다 심경 }

	\author{ 2020년 8월 14일 금요일 }

%	\institute{김대희}
%	\titlegraphic{\includegraphics[width=7cm]{IMG_1934}}

	%% Optional title graphic
	%\titlegraphic{\includegraphics[width=7cm]{IMG_1934}}
	%% Uncomment to switch off tikzposter footer
	% \tikzposterlatexaffectionproofoff

\begin{document}

	\maketitle





%	------------------------------------------------------------------------------ 神妙章句大陀羅尼 신묘장구대다라니
			\block{■  마하 반야 바라밀다 심경  }
			{
				\begin{LARGE}

관자재보살이 깊은 반야바라밀다를 행할 때,
오온이 공한 것을 비추어 보고 온갖 고통에서 건너느니라.

사리자여! 색이 공과 다르지 않고 공이 색과 다르지 않으며 ,
색이 곧 공이요 공이 곧 색이니, 수 상 행 식도 그러하니라.

사리자여! 모든법은 공하여 나지도 멸하지도 않으며 ,
더럽지도 깨끗하지도 않으며 , 늘지도 줄지도 않느니라.
그러므로 공 가운데는 색이 없고 수 상 행 식도 없으며 ,
안 이 비 설 신 의도 없고,
색 성 향 미 촉 법도 없으며 ,
눈의 경계도 의식의 경계까지도 없고,
무명도 무명이 다함까지도 없으며 ,
늙고 죽음도 늙고 죽음이 다함까지도 없고,
고 집 멸 도도 없으며 , 지혜도 얻음도 없느니라.
얻을 것이 없는 까닭에 보살은 반야바라밀다를 의지하므로
마음에 걸림이 없고 걸림이 없으므로 두려움이 없어서,
뒤바뀐 헛된 생각을 멀 리 떠나 완전한 열반에 들어가며 ,
삼세의 모든 부처님도 반야바라밀다를 의지하므로
최상의 깨달음을 얻느니라.

반야바라밀다는 가장 신비하고 밝은 주문이며 위없는 주문이며
무엇과도 견줄 수 없는 주문이니,
온갖 괴로움을 없애고 진실하여 허망 하지 않음을 알지니라.
이제 반야바라밀다주를 말하리라.

아제아제 바라아제 바라승아제 모지 사바하(3번)

				\end{LARGE}
			}




\end{document}


		\begin{huge}
		\end{huge}

		\begin{LARGE}
		\end{LARGE}

		\begin{Large}
		\end{Large}

		\begin{large}
		\end{large}

