%	------------------------------------------------------------------------------
%
%		작성 : 2020년 7월 13일 첫 작업 포교사 고시 끝나고 정리 중에 목차 정리 
%
%

%	\documentclass[25pt, a1paper]{tikzposter}
%	\documentclass[20pt, a0paper, landscape]{tikzposter}
%	\documentclass[25pt, a1paper ]{tikzposter}

	\documentclass[	20pt, 
							a0paper, 
%							portrait, %
							landscape,
							margin=0mm, %
							innermargin=10mm,  		%
							blockverticalspace=4mm, %
							colspace=5mm, 
							subcolspace=0mm
							]{tikzposter}

%	\documentclass[25pt, a1paper]{tikzposter}
%	\documentclass[25pt, a1paper]{tikzposter}
%	\documentclass[25pt, a1paper]{tikzposter}

% 	12pt  14pt 17pt  20pt  25pt
%
%	a0 a1 a2
%
%	landscape  portrait
%

	%% Tikzposter is highly customizable: please see
	%% https://bitbucket.org/surmann/tikzposter/downloads/styleguide.pdf

	%	========================================================== 	Package
		\usepackage{kotex}						% 한글 사용


%% Available themes: see also
%% https://bitbucket.org/surmann/tikzposter/downloads/themes.pdf
%	\usetheme{Default}
%	\usetheme{Rays}
%	\usetheme{Basic}
	\usetheme{Simple}
%	\usetheme{Envelope}
%	\usetheme{Wave}
%	\usetheme{Board}
%	\usetheme{Autumn}
%	\usetheme{Desert}

%% Further changes to the title etc is possible
%	\usetitlestyle{Default}			%
%	\usetitlestyle{Basic}				%
%	\usetitlestyle{Empty}				%
%	\usetitlestyle{Filled}				%
%	\usetitlestyle{Envelope}			%
%	\usetitlestyle{Wave}				%
%	\usetitlestyle{verticalShading}	%


%	\usebackgroundstyle{Default}
%	\usebackgroundstyle{Rays}
%	\usebackgroundstyle{VerticalGradation}
%	\usebackgroundstyle{BottomVerticalGradation}
%	\usebackgroundstyle{Empty}

%	\useblockstyle{Default}
%	\useblockstyle{Basic}
%	\useblockstyle{Minimal}		% 이것은 간단함
%	\useblockstyle{Envelope}		% 
%	\useblockstyle{Corner}		% 사각형
%	\useblockstyle{Slide}			%	띠모양  
	\useblockstyle{TornOut}		% 손그림모양


	\usenotestyle{Default}
%	\usenotestyle{Corner}
%	\usenotestyle{VerticalShading}
%	\usenotestyle{Sticky}

%	\usepackage{fontspec}
%	\setmainfont{FreeSerif}
%	\setsansfont{FreeSans}

%	------------------------------------------------------------------------------ 제목

	\title{ 볼교 정리 순서 }

	\author{ 2020년 7월 }

%	\institute{서영엔지니어링}
%	\titlegraphic{\includegraphics[width=7cm]{IMG_1934}}

	%% Optional title graphic
	%\titlegraphic{\includegraphics[width=7cm]{IMG_1934}}
	%% Uncomment to switch off tikzposter footer
	% \tikzposterlatexaffectionproofoff

\begin{document}
	\maketitle[
					width=841mm,
					linewidth = 2mm,
					innersep=4mm,
					titletotopverticalspace=2mm, %
					titletoblockverticalspace=2mm, %
					titletextscale =4, 
				]

		%		a0  841 - 1189
		%		a1  594 - 841
		%		a2  420 - 594

	\begin{columns}

	%	====== ====== ====== ====== ====== 
		\column{0.2}

%	------------------------------------------------------------------------------ 부처님의 일대기
			\block{■  부처님의  일대기  }
			{
					\setlength{\leftmargini}{4em}
					\setlength{\labelsep} {1em}
				\begin{LARGE}
					\begin{itemize}
					\item 부처님 가계도
					\item 인도 지역
					\item 성지
					\item 불교 명절
					\item 재일 
					\item 우란분절 백중
					\item 안거 

					\item 팔상도
					\item 전생
					\item 탄생
					\item 출가
					\item 수행
					\item 초전법륜
					\item 전도
					\item 말년
					\item 열반
					\item
					\item
					\end{itemize}
				\end{LARGE}
			} %	-----------------------------------------------------------------


%	------------------------------------------------------------------------------ 부처
			\block{■  부처  }
			{
					\setlength{\leftmargini}{4em}
					\setlength{\labelsep} {1em}
				\begin{LARGE}
					\begin{itemize}
					\item 과거칠불
					\item 칠불통계게
					\item 명호
					\item 
					\end{itemize}
				\end{LARGE}
			} %	-----------------------------------------------------------------


%	------------------------------------------------------------------------------ 보살
			\block{■  보살  }
			{
					\setlength{\leftmargini}{4em}
					\setlength{\labelsep} {1em}
				\begin{LARGE}
					\begin{itemize}
					\item 보살
					\item 대원
					\item 
					\end{itemize}
				\end{LARGE}
			} %	-----------------------------------------------------------------



%	------------------------------------------------------------------------------ 제자
			\block{■  제자  }
			{
					\setlength{\leftmargini}{4em}
					\setlength{\labelsep} {1em}
				\begin{LARGE}
					\begin{itemize}
					\item 오비구 
					\item 10대 제자 
					\item 
					\end{itemize}
				\end{LARGE}
			} %	-----------------------------------------------------------------

	%	====== ====== ====== ====== ====== 
		\column{0.2}

%	------------------------------------------------------------------------------ 결집
			\block{■  결집 }
			{
					\setlength{\leftmargini}{4em}
					\setlength{\labelsep} {1em}
				\begin{LARGE}
					\begin{itemize}
					\item 결집
					\item 삼장
					\item 팔리 경장
					\item 부파 불교
					\item 설일체유부
					\item 
					\item 
					\end{itemize}
				\end{LARGE}
			} %	-----------------------------------------------------------------



%	------------------------------------------------------------------------------ 경전
			\block{■  경전  }
			{
					\setlength{\leftmargini}{3em}
					\setlength{\labelsep} {1em}
				\begin{LARGE}3
					\begin{itemize}
					\item 경전의 분류
					\item 경전의 번역
					\item 

					\item 아함경
					\item 중아함경
					\item 잡아함경
					\item 


					\item 법화경 : 묘법연화경
					\item 화엄경 : 대방광불화엄경 
					\item 금강경 : 금강반야바라밀경
					\item 서장
					\item 전등법어
					\item  육조단경
					\item  마조록
					\item  임제록
					\item  벽암록
					\end{itemize}
				\end{LARGE}
			} %	-----------------------------------------------------------------

%	------------------------------------------------------------------------------ 스님
			\block{■  스님  }
			{
					\setlength{\leftmargini}{4em}
					\setlength{\labelsep} {1em}
				\begin{LARGE}
					\begin{itemize}
					\item 스님
					\item 역경가
					\item 
					\end{itemize}
				\end{LARGE}
			} %	-----------------------------------------------------------------


%	------------------------------------------------------------------------------ 총림
			\block{■  총림  }
			{
					\setlength{\leftmargini}{4em}
					\setlength{\labelsep} {1em}
				\begin{LARGE}
					\begin{itemize}
					\item 총림
					\item 
					\item 
					\end{itemize}
				\end{LARGE}
			} %	-----------------------------------------------------------------


	%	====== ====== ====== ====== ====== 
		\column{0.2}


%	------------------------------------------------------------------------------ 불상
			\block{■  블상  }
			{
					\setlength{\leftmargini}{4em}
					\setlength{\labelsep} {1em}
				\begin{LARGE}
					\begin{itemize}
					\item 불상 
					\item 수인
					\item 계인
					\item 삼신불
					\item 보살
					\item 팔부신중
					\item 

					\end{itemize}
				\end{LARGE}
			} %	-----------------------------------------------------------------


%	------------------------------------------------------------------------------ 탑
			\block{■  탑  }
			{
					\setlength{\leftmargini}{4em}
					\setlength{\labelsep} {1em}
				\begin{LARGE}
					\begin{itemize}
					\item 
					\item 
					\item 
					\end{itemize}
				\end{LARGE}
			} %	-----------------------------------------------------------------



%	------------------------------------------------------------------------------ 건축
			\block{■  건축  }
			{
					\setlength{\leftmargini}{4em}
					\setlength{\labelsep} {1em}
				\begin{LARGE}
					\begin{itemize}
					\item 전각배치
					\item 
					\item 
					\end{itemize}
				\end{LARGE}
			} %	-----------------------------------------------------------------


%	------------------------------------------------------------------------------ 미술
			\block{■  미술  }
			{
					\setlength{\leftmargini}{4em}
					\setlength{\labelsep} {1em}
				\begin{LARGE}
					\begin{itemize}
					\item 불화
					\item 탱화
					\item 조각 
					\item 단청

					\end{itemize}
				\end{LARGE}
			} %	-----------------------------------------------------------------



	%	====== ====== ====== ====== ====== 
		\column{0.2}


%	------------------------------------------------------------------------------ 팔정도
			\block{■  팔정도  }
			{
					\setlength{\leftmargini}{4em}
					\setlength{\labelsep} {1em}
				\begin{LARGE}
					\begin{itemize}
					\item 
					\item 
					\item 
					\item 
					\end{itemize}
				\end{LARGE}
			}

%	------------------------------------------------------------------------------ 사성제
			\block{■  사성제  }
			{
					\setlength{\leftmargini}{4em}
					\setlength{\labelsep} {1em}
				\begin{LARGE}
					\begin{itemize}
					\item 
					\item 
					\item 
					\item 
					\end{itemize}
				\end{LARGE}
			}



	%	====== ====== ====== ====== ====== 
		\column{0.2}

%	------------------------------------------------------------------------------ 인도불교
			\block{■  인도 불교  }
			{
					\setlength{\leftmargini}{3em}
					\setlength{\labelsep} {1em}
				\begin{LARGE}
					\begin{itemize}
					\item 불교시대 구분
					\item 
					\item 
					\item 
					\end{itemize}
				\end{LARGE}
			} %	-----------------------------------------------------------------


%	------------------------------------------------------------------------------ 중국불교
			\block{■  중국 불교  }
			{
					\setlength{\leftmargini}{3em}
					\setlength{\labelsep} {1em}
				\begin{LARGE}
					\begin{itemize}
					\item 중국 선종
					\item 육조 혜능 
					\item 
					\end{itemize}
				\end{LARGE}
			} %	-----------------------------------------------------------------


%	------------------------------------------------------------------------------ 한국불교
			\block{■  한국 불교  }
			{
					\setlength{\leftmargini}{3em}
					\setlength{\labelsep} {1em}
				\begin{LARGE}
					\begin{itemize}
					\item 삼국
					\item 고려
					\item 조선
					\item 강점기
					\item 근세
					\item 현대

					\item 종파
					\item 조계종

					\item 사찰


					\end{itemize}
				\end{LARGE}
			} %	-----------------------------------------------------------------


%	------------------------------------------------------------------------------ 일본불교
			\block{■  일본 불교  }
			{
					\setlength{\leftmargini}{3em}
					\setlength{\labelsep} {1em}
				\begin{LARGE}
					\begin{itemize}
					\item 
					\item 
					\item 
					\end{itemize}
				\end{LARGE}
			} %	-----------------------------------------------------------------







	\end{columns}




\end{document}


		\begin{huge}
		\end{huge}

		\begin{LARGE}
		\end{LARGE}

		\begin{Large}
		\end{Large}

		\begin{large}
		\end{large}

