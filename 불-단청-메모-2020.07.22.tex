%	-------------------------------------------------------------------------------
%
%		작성		2020년 
%				7월 
%				22일 
%				수요일
%
%
%
%
%
%
%	-------------------------------------------------------------------------------

%	\documentclass[10pt,xcolor=pdftex,dvipsnames,table]{beamer}
%	\documentclass[10pt,blue,xcolor=pdftex,dvipsnames,table,handout]{beamer}
%	\documentclass[14pt,blue,xcolor=pdftex,dvipsnames,table,handout]{beamer}
	\documentclass[aspectratio=1610,10pt,xcolor=pdftex,dvipsnames,table,handout]{beamer}
%	\documentclass[aspectratio=169,17pt,xcolor=pdftex,dvipsnames,table,handout]{beamer}
%	\documentclass[aspectratio=149,17pt,xcolor=pdftex,dvipsnames,table,handout]{beamer}
%	\documentclass[aspectratio=54,17pt,xcolor=pdftex,dvipsnames,table,handout]{beamer}
%	\documentclass[aspectratio=43,17pt,xcolor=pdftex,dvipsnames,table,handout]{beamer}
%	\documentclass[aspectratio=32,17pt,xcolor=pdftex,dvipsnames,table,handout]{beamer}

		% Font Size
		%	default font size : 11 pt
		%	8,9,10,11,12,14,17,20
		%
		% 	put frame titles 
		% 		1) 	slideatop
		%		2) 	slide centered
		%
		%	navigation bar
		% 		1)	compress
		%		2)	uncompressed
		%
		%	Color
		%		1) blue
		%		2) red
		%		3) brown
		%		4) black and white	
		%
		%	Output
		%		1)  	[default]	
		%		2)	[handout]		for PDF handouts
		%		3) 	[trans]		for PDF transparency
		%		4)	[notes=hide/show/only]

		%	Text and Math Font
		% 		1)	[sans]
		% 		2)	[sefif]
		%		3) 	[mathsans]
		%		4)	[mathserif]


		%	---------------------------------------------------------	
		%	슬라이드 크기 설정 ( 128mm X 96mm )
		%	---------------------------------------------------------	
%			\setbeamersize{text margin left=2mm}
%			\setbeamersize{text margin right=2mm}

	%	========================================================== 	Package
		\usepackage{kotex}						% 한글 사용
		\usepackage{amssymb,amsfonts,amsmath}	% 수학 수식 사용
		\usepackage{color}					%
		\usepackage{colortbl}					%


	%		========================================================= 	note 옵션인 
	%			\setbeameroption{show only notes}
		

	%		========================================================= 	Theme

		%	---------------------------------------------------------	
		%	전체 테마
		%	---------------------------------------------------------	
		%	테마 명명의 관례 : 도시 이름
%			\usetheme{default}			%
%			\usetheme{Madrid}    		%
%			\usetheme{CambridgeUS}    	% -red, no navigation bar
%			\usetheme{Antibes}			% -blueish, tree-like navigation bar

		%	----------------- table of contents in sidebar
			\usetheme{Berkeley}		% -blueish, table of contents in sidebar
									% 개인적으로 마음에 듬

%			\usetheme{Marburg}			% - sidebar on the right
%			\usetheme{Hannover}		% 왼쪽에 마크
%			\usetheme{Berlin}			% - navigation bar in the headline
%			\usetheme{Szeged}			% - navigation bar in the headline, horizontal lines
%			\usetheme{Malmoe}			% - section/subsection in the headline

%			\usetheme{Singapore}
%			\usetheme{Amsterdam}

		%	---------------------------------------------------------	
		%	색 테마
		%	---------------------------------------------------------	
%			\usecolortheme{albatross}	% 바탕 파란
%			\usecolortheme{crane}		% 바탕 흰색
%			\usecolortheme{beetle}		% 바탕 회색
%			\usecolortheme{dove}		% 전체적으로 흰색
%			\usecolortheme{fly}		% 전체적으로 회색
%			\usecolortheme{seagull}	% 휜색
%			\usecolortheme{wolverine}	& 제목이 노란색
%			\usecolortheme{beaver}

		%	---------------------------------------------------------	
		%	Inner Color Theme 			내부 색 테마 ( 블록의 색 )
		%	---------------------------------------------------------	

%			\usecolortheme{rose}		% 흰색
%			\usecolortheme{lily}		% 색 안 칠한다
%			\usecolortheme{orchid} 	% 진하게

		%	---------------------------------------------------------	
		%	Outter Color Theme 		외부 색 테마 ( 머리말, 고리말, 사이드바 )
		%	---------------------------------------------------------	

%			\usecolortheme{whale}		% 진하다
%			\usecolortheme{dolphin}	% 중간
%			\usecolortheme{seahorse}	% 연하다

		%	---------------------------------------------------------	
		%	Font Theme 				폰트 테마
		%	---------------------------------------------------------	
%			\usfonttheme{default}		
			\usefonttheme{serif}			
%			\usefonttheme{structurebold}			
%			\usefonttheme{structureitalicserif}			
%			\usefonttheme{structuresmallcapsserif}			



		%	---------------------------------------------------------	
		%	Inner Theme 				
		%	---------------------------------------------------------	

%			\useinnertheme{default}
			\useinnertheme{circles}		% 원문자			
%			\useinnertheme{rectangles}		% 사각문자			
%			\useinnertheme{rounded}			% 깨어짐
%			\useinnertheme{inmargin}			




		%	---------------------------------------------------------	
		%	이동 단추 삭제
		%	---------------------------------------------------------	
%			\setbeamertemplate{navigation symbols}{}

		%	---------------------------------------------------------	
		%	문서 정보 표시 꼬리말 적용
		%	---------------------------------------------------------	
%			\useoutertheme{infolines}


			
	%	---------------------------------------------------------- 	배경이미지 지정
%			\pgfdeclareimage[width=\paperwidth,height=\paperheight]{bgimage}{./fig/Chrysanthemum.jpg}
%			\setbeamertemplate{background canvas}{\pgfuseimage{bgimage}}

		%	---------------------------------------------------------	
		% 	본문 글꼴색 지정
		%	---------------------------------------------------------	
%			\setbeamercolor{normal text}{fg=purple}
%			\setbeamercolor{normal text}{fg=red!80}	% 숫자는 투명도 표시


		%	---------------------------------------------------------	
		%	itemize 모양 설정
		%	---------------------------------------------------------	
%			\setbeamertemplate{items}[ball]
%			\setbeamertemplate{items}[circle]
%			\setbeamertemplate{items}[rectangle]






		\setbeamercovered{dynamic}





		% --------------------------------- 	문서 기본 사항 설정
		\setcounter{secnumdepth}{3} 		% 문단 번호 깊이
		\setcounter{tocdepth}{3} 			% 문단 번호 깊이




% ------------------------------------------------------------------------------
% Begin document (Content goes below)
% ------------------------------------------------------------------------------
	\begin{document}
	

			\title{ 불교 단청  }
			\author{ 김대희 }
			\date{ 2020년 7월 22일 수요일 }


% -----------------------------------------------------------------------------
%		개정 내용
% -----------------------------------------------------------------------------
%
%		2020년 6월 28일 첫제작
%
%
%


	%	==========================================================
	%
	%	----------------------------------------------------------
		\begin{frame}[plain]
		\titlepage
		\end{frame}


		\begin{frame} [plain]{목차}
		\tableofcontents%
		\end{frame}



	%	========================================================== 책자
		\part{ 책자 }
		\frame{\partpage}

		\begin{frame} [plain]{목차}
		\tableofcontents%
		\end{frame}
		

	%	---------------------------------------------------------- 책자
	%		Frame
	%	----------------------------------------------------------
		\section{ 책자 }
		\begin{frame} [t,plain]
		\frametitle{ 책자 }
			\begin{block} { 책자 }
			\setlength{\leftmargini}{2em}			
			\begin{itemize}
				\item (우리가 정말 알아야 할) 우리 단청
				\item 단청 금문디자인=Traditional pattern of Korea Dancheong : 한국전통문양의 맥
				\item 단청 예찬
				\item 단청 장
				\item 불사 및 기도 안내문 모음집
				\item 빛깔있는 책들 : 고미술. 제24권 : 단청
				\item 사찰장식 그 빛나는 상징의 세계
				\item 아름다운 우리 무늬. 2, 기와.나비.단청 .장석
				\item 알기쉬운 불교미술
				\item 예술로서의 단청 =Dancheong as art
				\item 오색빛을 찾아서
				\item 중국건축도해사전 = Illustration dictionary of classical Chinese architecture
				\item 한국 산사의 단청 세계 : 불교건축에 펼친 화엄의 빛
				\item 한국단청 의 원류 : 발생에서 고려시대까지
				\item 한국의 단청
			\end{itemize}
			\end{block}						

		\end{frame}						
		

	%	---------------------------------------------------------- 한국 산사의 단청 세계 : 불교건축에 펼친 화엄의 빛 }
	%		Frame
	%	----------------------------------------------------------
		\section{ 한국 산사의 단청 세계 : 불교건축에 펼친 화엄의 빛 } 
		\begin{frame} [t,plain]
		\frametitle{ 한국 산사의 단청 세계 : 불교건축에 펼친 화엄의 빛 }
			\begin{block} {  한국 산사의 단청 세계 : 불교건축에 펼친 화엄의 빛 }
			\setlength{\leftmargini}{4em}			
			\begin{itemize}
				\item [제목] 한국 산사의 단청 세계 : 불교건축에 펼친 화엄의 빛 
				\item [지은이] 노재학
				\item [도서관] 
구덕 청구기호 : 619-15
남구 청구기호 : 619-노73한


				\item 1부	중중무진의 연화장 세계
				\item 2부	법자, 경전, 불보살의 단청장엄
				\item 3부	단청 벽화의 향연
				\item 4부	주악비천과 공양비천
				\item 5부	용, 봉황, 선학의 상서
				\item 6부	태극, 별자리, 우주 천문의 단청 세계
				\item 7부	건축 뼈대의 윤리 미
				\item 부록	오래된 벽화 및 단청문양이 현존하는 사찰건물 지역별 분포현황



			\end{itemize}
			\end{block}			

								
		\end{frame}						
	

	%	---------------------------------------------------------- 법당 단청의 표현원리 }
	%		Frame
	%	----------------------------------------------------------
		\section{ 법당 단청의 표현원리 }
		\begin{frame} [t,plain]
			\begin{block} { 법당 단청의 표현원리 }
			\setlength{\leftmargini}{6em}			
			\begin{itemize}
				\item 	생명을 화생시키는 자비의 표현 	:	연꽃, 넝쿨문양
				\item 	무명을 밝히는 법계의 진리 표현	:	보주, 광명, 법자진어
				\item 	법계 수호와 신성한 에너지의 표현	:	용, 봉황
				\item 	불보살에 헌공하는 공양장엄	:	꽃, 주악비천과 공양비천

			\end{itemize}
			\end{block}						
								
		\end{frame}						

	%	---------------------------------------------------------- 공양주보살
	%		Frame
	%	----------------------------------------------------------
		\section{공양주 보살}
		\begin{frame} [t,plain]
		\frametitle{공양주 보살}
			\begin{block} {공양주 보살}
			\setlength{\leftmargini}{6em}			
			\begin{itemize}
				\item [공양주 보살] 
			\end{itemize}
			\end{block}						
								
		\end{frame}						

	%	---------------------------------------------------------- 법당보살
	%		Frame
	%	----------------------------------------------------------
		\section{법당 보살}
		\begin{frame} [t,plain]
		\frametitle{법당 보살}
			\begin{block} {법당 보살}
			\setlength{\leftmargini}{6em}			
			\begin{itemize}
				\item [법당 보살] 
			\end{itemize}
			\end{block}						
								
		\end{frame}						


	%	---------------------------------------------------------- 부회장

	%	---------------------------------------------------------- 회장
	%		Frame
	%	----------------------------------------------------------
		\section{신도회 회장}
		\begin{frame} [t,plain]
		\frametitle{신도회 회장}
			\begin{block} {신도회 회장}
			\setlength{\leftmargini}{6em}			
			\begin{itemize}
				\item [신도회 회장] 
			\end{itemize}
			\end{block}						
								
		\end{frame}						



			

	%	========================================================== 강의
		\part{강의}
		\frame{\partpage}
		
		\begin{frame} [plain]{목차}
		\tableofcontents%
		\end{frame}
		

% -----------------------------------------------------------------------------
%
% -----------------------------------------------------------------------------
	\section{서장 }
	\frame [plain] {\sectionpage}


% -----------------------------------------------------------------------------
%
% -----------------------------------------------------------------------------
	\section{육조단경 }
	\frame [plain] {\sectionpage}


% -----------------------------------------------------------------------------
%
% -----------------------------------------------------------------------------
	\section{법화경 }
	\frame [plain] {\sectionpage}

% -----------------------------------------------------------------------------
%
% -----------------------------------------------------------------------------
	\section{  }
	\frame [plain] {\sectionpage}



	%	========================================================== 법회
		\part{법회}
		\frame{\partpage}

		\begin{frame} [plain]{목차}
		\tableofcontents%
		\end{frame}


% -----------------------------------------------------------------------------
%
% -----------------------------------------------------------------------------
	\section{ 초하루  법회}
	\frame [plain] {\sectionpage}

% -----------------------------------------------------------------------------
%
% -----------------------------------------------------------------------------
	\section{ 보름  법회}
	\frame [plain] {\sectionpage}


	%	========================================================== 기도
		\part{기도}
		\frame{\partpage}

%		\begin{frame} [plain]{목차}
%		\tableofcontents%
%		\end{frame}


% -----------------------------------------------------------------------------
%
% -----------------------------------------------------------------------------
		\section{다라니기도}
		\begin{frame} [t,plain]
		\frametitle{다라니기도}
			\begin{block} {다라니기도 }
			\setlength{\leftmargini}{5em}			
			\begin{itemize}
				\item [장소]	 법당
				\item [일시]	 매주 수요일 저녁 8시
				\item [금액]	
			\end{itemize}
			\end{block}						
		\end{frame}					






	%	========================================================== 행사
		\part{행사}
		\frame{\partpage}
		
		\begin{frame} [plain]{목차}
		\tableofcontents%
		\end{frame}
		

	

	

	%	---------------------------------------------------------- 대중공양
	%		Frame
	%	----------------------------------------------------------
		\section{대중공양}
		\begin{frame} [t,plain]
		\frametitle{대중공양}
			\begin{block} {대중공양 }
			\setlength{\leftmargini}{5em}			
			\begin{itemize}
				\item [장소]	 강원도 상원사
				\item [일시]	7월 16일 (목) 오전 7시 출발
				\item [금액]	
			\end{itemize}
			\end{block}						
		\end{frame}					

	%	---------------------------------------------------------- 백중기도 
	%		Frame
	%	----------------------------------------------------------
		\section{백중기도 및 영가천도 49재}
		\begin{frame} [t,plain]
		\frametitle{백중기도 및 영가천도 49재 }
			\begin{block} {백중기도 및 영가천도 49재 }

			\setlength{\leftmargini}{5em}			
			\begin{itemize}
				\item [입재]	 강원도 상원사
				\item [회향]	7월 16일 (목) 오전 7시 출발
				\item [망축기도]	영가1위 1만원
				\item [생축기도]	가족당 3만원
			\end{itemize}

			\end{block}						
		\end{frame}					




% ------------------------------------------------------------------------------
% End document
% ------------------------------------------------------------------------------





\end{document}


